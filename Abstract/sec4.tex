\section{その他}
その他の事項として,本節では表の記述方法と参考文献について記載する.

\subsection{表の記述}
論文を記述する際にも指摘したが,表においては数値は右詰にしなければならない.また,ラベル部は中央揃えとすることが多い.

そのような設定をしたものを 表~\ref{table:face_rec}に示す.

\begin{table}[b]
\centering
\caption{WHLACによる顔表情認識率}
\label{table:face_rec}
\vspace{2mm}
\small
\begin{tabular}{|r|r|r|r|} \hline
\multicolumn{1}{|c|}{Data \#} & \multicolumn{1}{c|}{Ave.} & \multicolumn{1}{c|}{Max.} & \multicolumn{1}{c|}{Min.} \\ \hline\hline
1 &  0.67 (N/A) & 0.91 (39) & 0.46(21) \\ \hline
2 & 0.37 (N/A) & 0.50 (38) & 0.09(10) \\ \hline
3 & 0.65 (N/A) & 0.87 (45) & 0.28(10) \\ \hline\hline
\multicolumn{1}{|c|}{Total Ave.} & 0.56 & 0.76 & 0.27 \\ \hline
\end{tabular}
\end{table} %

\subsection{参考文献について}
抄録においては,参考文献のフォーマットも省略することが多いのだが,今回は論文時と同様の表記にて提出することとした.

参考文献を記載するファイルは新たに作成せず,論文と同じ \verb+myrefs.bib+ファイルをスタイルパッケージのフォルダにコピーし,しかるべき引用命令を入れれば良い.
サンプルとして,論文\cite{Kogami2009},書籍(の一部)\cite{WelfareJapan},書籍\cite{Nakata2010},予稿集\cite{Miyaji2003ROMAN},その他(Webサイトなど)\cite{HTUlatex}を組み込んだ.
