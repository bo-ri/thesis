\section{抄録を作成する際の注意事項}
論文抄録とは,簡易なことばで表現すると「論文を要約して書き出したもの」である.
同じように論文を要約した者としては,論文要旨が存在する.
宮治研では節や図表などの論文の体裁に近い形でのものを「抄録」と,節や図表などの情報が記載されていないものを「要旨」とよんで区別している.
社会情報学部では,宮治研でいうところの論文抄録を,論文要旨ということばで指示することがあるため,注意が必要である.

抄録は,論文と同じような見た目ではあるが,その要約の関係から論文の章や節の構成が異なる(ことが多い).
また,論文とは異なる書き方のルールが存在することに注意が必要である.
特に,抄録内の図表は,そのページの上または下に載せる.
ここで,図表が複数存在する場合には,上下に分散して配置されても良いし,それらが縦方向にまとめて並べられてもよい.

以上の点については,Word・\LaTeX のどちらで抄録を作成するにせよ,守る必要がある.

卒業研究の抄録(要旨)の Wordの雛形ファイルおよび \LaTeX のスタイルパッケージは Dropbox上に配置した.
同じフォルダ上に置いた仕上がりのPDFイメージを確認してから作業をすること.また,\LaTeX にて文書を作成する者は,本文書の残りの部分の利用方法および注意事項に目をとおすこと.