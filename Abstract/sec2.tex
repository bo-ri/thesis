\section{宮治研 抄録\LaTeX スタイルパッケージの使い方}
基本的に論文のスタイルパッケージと同様に作業をすれば良い.
例えば,\verb+main.tex+ファイルに必要事項を記載し,適切なファイルを取り込むように指定し,バッチコマンドを利用すれば,PDFファイルが出来上がる.

なお,抄録を記述する際注意事項として,スタイルパッケージの利用方法以外については次節にて解説する.

\subsection{サブタイトル有りの場合}
配布したファイルは,サブタイトルがある場合のサンプルになっている.
各自の 年度,学籍番号,氏名,タイトル,サブタイトルを所定の命令内に記入する.
\begin{screen}
{\small
%footnotesize
\begin{verbatim}
\nendo{2013年度}
\snum{15387019}
\jname{宮治 裕}
\thesistitle{宮治研における論文作成について}
\thesissubtitle{\LaTeX の利用}
\SUBTtrue
%\SUBTfalse
\end{verbatim}
}
\end{screen}

\subsection{サブタイトル無しの場合}
サブタイトル有りの場合と比較して3箇所の変更が必要である.
\begin{enumerate}
\item サブタイトルを記入する命令の先頭部分に \verb+%+ 記号を入れ,コメントアウト状態にする
\item \verb+\SUBTtrue+の前に\verb+%+ 記号を入れ,コメントアウト状態にする
\item コメントアウト状態の \verb+\SUBTfalse+の直前の \verb+%+ 記号を削除する
\end{enumerate}
以上の変更を行った設定を,以下に示す.
\begin{screen}
{\small
%footnotesize
\begin{verbatim}
%\thesissubtitle{}
%\SUBTtrue
\SUBTfalse
\end{verbatim}
}
\end{screen}
