\section{GloVeモデルの考察}
GloVeで生成したモデルで出力した3つの結果それぞれの観点からモデルの精度を考察する.

\subsection{青山学院大学と近い大学}
このタスクにおいて最も妥当性が高いと考えられるモデルは,GV\_ C 表 \ref{table:gvc}である.
出力結果から,立教大学,中央大学,明治大学,法政大学が取得できた.
また,ミッション系の大学で上智大学,偏差値が近いと考えられる東京理科大学や成蹊大学,学習院大学が得られた.

GV\_A 表 \ref{table:gva}とGV\_ B 表 \ref{table:gvb},GV\_ D 表 \ref{table:gvd}も出力結果に大きな差はなかった.
これらの結果から,モデルのwindowサイズは10が妥当であることが分かる.
おそらくwindowサイズを1000に設定すると,学習が収束する前に終わってしまうため,適切な関係性を学習しきれなかったと考えられる.

\subsection{$ 青山学院大学 - キリスト教 $}
このタスクにおいて最も妥当性が高いと考えられるモデルは,WV\_ D 表 \ref{table:gvd}である.
理由としてはまず出力結果にミッション系の大学が含まれていない点が第1に挙げられる.
第2に中央大学,法政大学,明治大学が高い類似度で示されている点が挙げられる.

このほかではGV\_ C 表 \ref{table:gvc}が挙げられるが,上智大学と立教大学が出力されている点でキリスト教の減算がうまく機能していないと考えられる.
これは単語ベクトルのサイズの違いが影響していると考えられるが,これらの結果から,単語ベクトルのサイズは100次元が妥当であり,windowサイズは10で十分であると言える.

\subsection{青山学院大学と明治学院大学で共通の近い単語}
GloVeで生成したモデルでは,windowサイズが小さい場合,このタスクの結果が得られなかった.
最も妥当性の高いと考えられるモデルはGV\_H 表 \ref{table:gvh}である.
出力から,聖書,チャペル,キリスト,キリスト教などのミッション系の大学を連想させる単語と,統合や定期などの歴史的背景を連想させる単語が得られた.
他のモデルでは十分な出力を得られなかったため,このモデルが今回の検証実験で最も妥当性が高いと言える.
