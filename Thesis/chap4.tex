\chapter{検証実験}
モデルの精度がどの程度妥当かを検証するために,本章ではパラメータを微調整したモデルの出力結果をまとめる.
またそれぞれの出力結果に関して評価する.

\section{実験概要}
検証実験では,2通りの方法で生成したモデルを用いて出力を得た.
1つはWord2Vecで生成したモデルで,もう一方はGloVeで生成したモデルである.
また,それぞれの方法でモデルを生成する際に,異なるパラメータを適用していくつかのモデルを学習した.
適用したパラメータの詳細は次節で説明する.

出力した内容は,モデルに対して大学名を入力とし,入力された大学に近い大学名を出力とした.
また,入力と出力の2つの大学間において近さを定義する共通の単語について,それぞれの大学からの距離を示した.
例として,A大学に近い大学としてB大学が得られた場合,A大学とB大学で共通して近い単語を探す.
得られた単語からのそれぞれの大学間の距離が近ければ,それぞれの大学が近い要因としての単語を得ることができる.

モデルを用いて出力する内容は,青山学院大学に近い大学名, $ 青山学院大学 - キリスト教 $ に該当する大学名,青山学院大学と明治学院大学それぞれで共通する意味が近い単語の3項目とした.
明治学院大学と比較した理由は,同じミッション系の大学で神学部の統合を通した日本神学校の創立などの関係性を持っているためである.

\section{評価軸}
モデルを評価する際に,それぞれの出力に対して評価軸を定めた.

青山学院大学に近い大学に関する評価項目は,まず偏差値が近いこと,ミッション系の大学であること,キャンパスの立地の近さである.
本研究で参考にした偏差値とキャンパスの立地はパスナビ\cite{passNavi}を参考とした.
モデルに学部,学科の情報は考慮されていないため,キャンパスの近さは学部を考慮しない.

次に,$ 青山学院大学 - キリスト教 $ の評価項目は,非ミッション系の大学であること,偏差値が近いこと,キャンパスの立地が近いこととした.
この場合,大学の要素からはキリスト教が消えていることが期待されるため,非ミッション系の大学であることをもっとも重要な評価項目とする.

最後に,青山学院大学と明治学院大学で共通の近い単語の評価項目は,直接的な関係性を示す単語とした.
これは,両校がミッション系であることから,キリスト教に関する単語などがあげられる.
また,歴史的な背景から日本神学校(現 東京神学大学)の創設に両校の神学部が統合したことから,神学部に関する単語も考慮する.

\begin{table}[htbp]
\caption{各出力結果の評価軸}
\centering
\begin{tabular}{|l|l|l|}
\hline
青山学院大学に近い大学 & $ 青山学院大学 - キリスト教 $ & 青山学院大学と明治学院大学で共通の近い単語
\\ \hline \hline
偏差値が近い& 非ミッション系 & 直接的な関係性を示す単語 \\
ミッション系& 偏差値が近い & \\
立地 & 立地 & \\ \hline
\end{tabular}
\label{table:eval}
\end{table}

\section{パラメータの詳細}
各パラメータの詳細について解説する.

\subsection{単語ベクトルの次元数}
% word2vecとGloVeで単語ベクトルを学習する際に指定するsizeオプションでは,隠れ層の単語ベクトルの次元数を指定する.
% このオプションで指定したサイズ * 全体の単語数のサイズのベクトルに全ての単語を圧縮し,分散表現を得る.
% この次元数が大きすぎると効率的な分散表現を学習できないが,次元数が小さすぎると単語の特徴を十分にとらえきれなくなり,学習に時間を要する.
Word2VecとGloVeで指定する次元数は,小さすぎると単語の特徴を効率的に学習できず,大きすぎると適切な分散表現が学習できない.
一般的に,50 ~ 300次元を指定する.
本研究で使用したデータセットは比較的サイズが小さいため,単語ベクトルの次元数は50次元とした.

\subsection{反復回数}
% iterオプションでは,学習の反復回数を指定する.
% 反復回数が少ないと,最適な分散表現が得られる前に学習が終了してしまう.
% また,反復回数を増やすと学習に要する時間が増加する.
% 検証実験では,最適な反復回数として,100 ~ 1000回を比較対象とする.
Word2VecとGloVeのトレーニングの反復回数を指定する.
この数字の大きさに比例して学習に要する時間も大きくなる.
また,反復回数が少なすぎると十分に単語の特徴を学習できないため,検証実験では10, 100, 1000のパラメータで比較する.

\subsection{windowサイズ}
% ある単語の単語ベクトルを学習する際に,文書に出現した学習対象の単語から指定した単語数まで離れた単語を対象として学習する.
% windowオプションではこの単語数を指定する.
% 本研究で用いた学習データは,大学に関するプレス記事を用いたため,windowサイズは1000とした.
% これは,記事の中に出現する大学名の単語ベクトルを学習する際,対象となる単語は記事全体に出現すると考えられるためである.
windowサイズは10と1000で比較した.
一般的にはwindowサイズは10 ~ 20で学習するが,記事の中に出現する大学名の単語ベクトルを学習する際,対象となる単語は記事全体に出現すると考えられるためwindowサイズに1000を適用して比較する.

\subsection{x-max}
GloVeの学習を行う際に,共起頻度の閾値を指定する必要がある.
Jeffrey Penningtonらの実験では,100,000,000 ~ 600,000,000個のトークンが含まれたコーパスを用いて,x-maxオプションに100を指定した.
一方本論文で学習したデータは約2,400,000個のトークンが含まれたデータを用いたため,x-maxオプションに指定する値は10と5で比較する.


\section{Word2Vecのモデル検証結果}
Word2Vecのモデル検証結果を示す.
今回検証したモデル名と,対応するパラメータの一覧を表 \ref{table:wvResultAll}に示す.

\begin{table}[htbp]
\caption{Word2Vecパラメータの詳細}
\centering
\begin{tabular}{llll}
\hline
モデル名 & 単語ベクトルの次元数 & 反復回数 & windowサイズ
\\ \hline \hline
WV\_ A & 50 & 10 & 10\\ \hline
WV\_ B & 100 & 10 & 10\\ \hline
WV\_ C & 50 & 100 & 10\\ \hline
WV\_ D & 100 & 100 & 10\\ \hline
WV\_ E & 50 & 10 & 1000\\ \hline
WV\_ F & 100 & 10 & 1000\\ \hline
WV\_ G & 50 & 100 & 1000\\ \hline
WV\_ H & 100 & 100 & 1000\\ \hline
\end{tabular}
\label{table:wvResultAll}
\end{table}

\subsection{WV\_ A}
単語ベクトルの次元数が50次元,反復回数とwindowサイズがそれぞれ10で生成したモデルの結果を表 \ref{table:wva}に示す.

青山学院大学に2番目に近い大学として明治学院大学が出現したが,それ以外の結果は関連性が不透明である.
この結果からは有効なモデルだと判断できない.

\begin{table}[H]
\caption{WV\_ Aの検証結果}
\centering
\footnotesize
% \scriptsize
\begin{tabular}{ll|ll|ll}
\hline
\multicolumn{2}{c}{青山学院大学に近い大学} & \multicolumn{2}{c}{青山学院大学 - キリスト教} & \multicolumn{2}{c}{青山学院大学と明治学院大学で共通の近い単語}
% \multicolumn{2}{c}{} & \\ \hline
\\ \hline
% 大学名 & 類似度 & 大学名 & 類似度 & 単語 & 青山学院大学との類似度 & 明治学院大学との類似度
大学名 & 類似度 & 大学名 & 類似度 & 単語 & 類似度の合計
\\ \hline \hline
聖心女子大学 & 0.74 & 目白大学 & 0.62 & 女学院 & 1.094\\
明治学院大学 & 0.722 & 帝京大学 & 0.583 & 英和 & 0.968\\
昭和女子大学 & 0.704 & 芝浦工業大学 & 0.552 & 芸術 & 0.742\\
帝京大学 & 0.703 & 女子美術大学 & 0.546 & ライン & 0.634\\
清泉女子大学 & 0.691 & 東京薬科大学 & 0.545 & 山手 & 0.579\\
実践女子大学 & 0.649 & 東京家政大学 & 0.544 & 校友 & 0.439\\
白百合女子大学 & 0.649 & 東京情報大学 & 0.542 & & \\
津田塾大学 & 0.638 & 東京電機大学 & 0.529 & & \\
東京女子体育大学 & 0.637 & 東京理科大学 & 0.525 & & \\
女子美術大学 & 0.636 & 多摩美術大学 & 0.524 & & \\ \hline
\end{tabular}
\label{table:wva}
\end{table}



\subsection{WV\_ B}
単語ベクトルの次元数が100次元,反復回数とwindowサイズがそれぞれ10で生成したモデルの結果を\ref{table:wvb}に示す.

青山学院大学に近い大学として,ミッション系の大学で偏差値も比較的近いと考えられる上智大学が得られた.
また $ 青山学院大学 - キリスト教 $ の結果から上位10校にはミッション系の大学が含まれなかった.
さらに,青山学院大学と明治学院大学で共通の近い意味の単語から,キリスト教などの単語が得られた.

しかし青山学院大学に近い大学の順位を全体的に見ると,関連性があると断言できない.
\begin{table}[H]
\caption{WV\_ Bの検証結果}
\centering
\footnotesize
\begin{tabular}{ll|ll|ll}
\hline
\multicolumn{2}{c}{青山学院大学に近い大学} & \multicolumn{2}{c}{青山学院大学 - キリスト教} & \multicolumn{2}{c}{青山学院大学と明治学院大学で共通の近い単語}
% \multicolumn{2}{c}{} & \\ \hline
\\ \hline
大学名 & 類似度 & 大学名 & 類似度 & 単語 & 類似度の合計
\\ \hline \hline
上智大学 & 0.372 & 成蹊大学 & 0.317 & 定期 & 0.531\\
東京外国語大学 & 0.308 & 駒澤大学 & 0.293 & 神学 & 0.45\\
駒澤大学 & 0.308 & 獨協大学 & 0.262 & 芸術 & 0.415\\
東京大学 & 0.295 & 東京理科大学 & 0.253 & 監督 & 0.37\\
筑波大学 & 0.294 & 中央大学 & 0.252 & キリスト & 0.364\\
早稲田大学 & 0.289 & 名古屋大学 & 0.221 & イギリス & 0.363\\
中央大学 & 0.276 & 筑波大学 & 0.204 & キリスト教 & 0.323\\
オックスフォード大学 & 0.271 & 明治大学 & 0.19 & 合同 & 0.304\\
東京農業大学 & 0.256 & 大阪大学 & 0.167 & 前期 & 0.291\\
成蹊大学 & 0.256 & 関西学院大学 & 0.167 & チャペル & 0.272\\ \hline
\end{tabular}
\label{table:wvb}
\end{table}


\subsection{WV\_ C}
単語ベクトルの次元数が50次元,反復回数が100,windowサイズを10で生成したモデルの結果を\ref{table:wvc}に示す.

青山学院大学に近い大学で上智大学や明治学院大学などが出現したが,単語ベクトルの類似度の高さでは清泉女子大学と聖心女子大学よりも低い.
どちらもミッション系の大学であるが,偏差値などの観点から上智大学や明治学院大学の方が近いと考えられる.

青山学院大学と明治学院大学で共通の近い単語では,神学という単語が出現した.
しかし,それぞれの大学との類似度の合計は他の単語に対して高くないため,学習不足であると考えられる.

\begin{table}[H]
\caption{WV\_ Cの検証結果}
\centering
\footnotesize
\begin{tabular}{ll|ll|ll}
\hline
\multicolumn{2}{c}{青山学院大学に近い大学} & \multicolumn{2}{c}{青山学院大学 - キリスト教} & \multicolumn{2}{c}{青山学院大学と明治学院大学で共通の近い単語}
% \multicolumn{2}{c}{} & \\ \hline
\\ \hline
大学名 & 類似度 & 大学名 & 類似度 & 単語 & 類似度の合計
\\ \hline \hline
清泉女子大学 & 0.677 & 東京薬科大学 & 0.557 & 女学院 & 0.981\\
聖心女子大学 & 0.674 & 帝京大学 & 0.539 & 英和 & 0.893\\
上智大学 & 0.633 & 中央大学 & 0.525 & 芸術 & 0.819\\
明治学院大学 & 0.616 & 目白大学 & 0.516 & ライン & 0.625\\
帝京大学 & 0.599 & 東京情報大学 & 0.498 & 学位 & 0.538\\
大東文化大学 & 0.585 & 女子美術大学 & 0.472 & 山手 & 0.538\\
立教大学 & 0.58 & 桜美林大学 & 0.469 & 校友 & 0.515\\
実践女子大学 & 0.578 & 東京電機大学 & 0.46 & 神学 & 0.463\\
桜美林大学 & 0.563 & 成蹊大学 & 0.458 & & \\
昭和女子大学 & 0.561 & 工学院大学 & 0.45 & & \\ \hline
\end{tabular}
\label{table:wvc}
\end{table}


\subsection{WV\_ D}
単語ベクトルの次元数が100次元,反復回数が100,windowサイズを10で生成したモデルの結果を\ref{table:wvd}に示す.

青山学院大学に近い大学として,上智大学や中央大学,学習院大学等が得られた.
しかし類似度が高い大学で仏教系大学である駒澤大学などが出現した.
青山学院大学と明治学院大学で共通の近い単語はキリストや礼拝,教会などミッション系の大学を連想させるような単語が新しく出現した.


\begin{table}[H]
\caption{WV\_ Dの検証結果}
\centering
\footnotesize
\begin{tabular}{ll|ll|ll}
\hline
\multicolumn{2}{c}{青山学院大学に近い大学} & \multicolumn{2}{c}{青山学院大学 - キリスト教} & \multicolumn{2}{c}{青山学院大学と明治学院大学で共通の近い単語}
% \multicolumn{2}{c}{} & \\ \hline
\\ \hline
大学名 & 類似度 & 大学名 & 類似度 & 単語 & 類似度の合計
\\ \hline \hline
上智大学 & 0.393 & 駒澤大学 & 0.337 & キリスト & 0.493\\
駒澤大学 & 0.364 & 中央大学 & 0.314 & 定期 & 0.491\\
東京外国語大学 & 0.304 & 獨協大学 & 0.261 & キリスト教 & 0.441\\
中央大学 & 0.3 & 関西学院大学 & 0.244 & 神学 & 0.433\\
東京農業大学 & 0.288 & 東洋大学 & 0.224 & 礼拝 & 0.424\\
関西学院大学 & 0.283 & 成蹊大学 & 0.208 & 宗教 & 0.416\\
早稲田大学 & 0.254 & 名古屋大学 & 0.178 & 芸術 & 0.39\\
筑波大学 & 0.253 & 明治大学 & 0.177 & 教会 & 0.358\\
獨協大学 & 0.253 & 筑波大学 & 0.166 & 基本 & 0.34\\
学習院大学 & 0.249 & 法政大学 & 0.165 & チャペル & 0.311\\ \hline
\end{tabular}
\label{table:wvd}
\end{table}

\subsection{WV\_ E}
単語ベクトルの次元数が50次元,反復回数が10,windowサイズを1000で生成したモデルの結果を\ref{table:wve}に示す.

このモデルでは,青山学院大学に近い大学として中央大学,法政大学,立教大学,明治大学が類似度の高い大学としてあげられた.
また,$ 青山学院大学 - キリスト教 $ はミッション系の大学が出現せずに法政大学や明治大学を得ることができた.

しかし,青山学院大学と明治学院大学で共通の近い単語から,神学やミッション系の大学に関する単語が出現しなくなった.

\begin{table}[H]
\caption{WV\_ Eの検証結果}
\centering
\footnotesize
\begin{tabular}{ll|ll|ll}
\hline
\multicolumn{2}{c}{青山学院大学に近い大学} & \multicolumn{2}{c}{青山学院大学 - キリスト教} & \multicolumn{2}{c}{青山学院大学と明治学院大学で共通の近い単語}
% \multicolumn{2}{c}{} & \\ \hline
\\ \hline
大学名 & 類似度 & 大学名 & 類似度 & 単語 & 類似度の合計
\\ \hline \hline
東洋大学 & 0.69 & 法政大学 & 0.558 & 芸術 & 0.787\\
日本女子大学 & 0.651 & 明治大学 & 0.527 & 音楽 & 0.741\\
聖心女子大学 & 0.636 & 立命館大学 & 0.49 & イギリス & 0.732\\
立命館大学 & 0.623 & 東洋大学 & 0.482 & 女学院 & 0.714\\
昭和女子大学 & 0.623 & 昭和女子大学 & 0.465 & バス & 0.711\\
中央大学 & 0.607 & 千葉工業大学 & 0.463 & コミュニティ & 0.643\\
法政大学 & 0.601 & 横浜市立大学 & 0.459 & 相談 & 0.601\\
立教大学 & 0.595 & 中央大学 & 0.455 & & \\
明治大学 & 0.584 & 金沢工業大学 & 0.455 & & \\
東京電機大学 & 0.569 & 神奈川大学 & 0.433 & & \\ \hline
\end{tabular}
\label{table:wve}
\end{table}

\subsection{WV\_ F}
単語ベクトルの次元数が100次元,反復回数が10,windowサイズを1000で生成したモデルの結果を\ref{table:wvf}に示す.

青山学院大学に近い大学は近い偏差値の大学やミッション系の大学が多く得られた.
青山学院大学と明治学院大学で共通の近い単語から,商業という単語が新しく得られた.
これは1944年に専門部を閉鎖し,明治学院に合同した際の高等商業学部から関連性があると考えられる.

\begin{table}[H]
\caption{WV\_ Fの検証結果}
\centering
\footnotesize
\begin{tabular}{ll|ll|ll}
\hline
\multicolumn{2}{c}{青山学院大学に近い大学} & \multicolumn{2}{c}{青山学院大学 - キリスト教} & \multicolumn{2}{c}{青山学院大学と明治学院大学で共通の近い単語}
% \multicolumn{2}{c}{} & \\ \hline
\\ \hline
大学名 & 類似度 & 大学名 & 類似度 & 単語 & 類似度の合計
\\ \hline \hline
明治大学 & 0.729 & 明治大学 & 0.426 & 心理 & 0.857\\
北里大学 & 0.713 & 法政大学 & 0.423 & 併設 & 0.85\\
法政大学 & 0.707 & 北里大学 & 0.385 & 統合 & 0.843\\
明治学院大学 & 0.66 & 九州大学 & 0.362 & 商業 & 0.817\\
短期大学 & 0.658 & 立命館大学 & 0.341 & 前期 & 0.794\\
上智大学 & 0.654 & 名古屋大学 & 0.337 & キリスト教 & 0.791\\
立教大学 & 0.65 & 駒澤大学 & 0.331 & イギリス & 0.72\\
中央大学 & 0.6 & 大阪大学 & 0.321 & 教会 & 0.679\\
早稲田大学 & 0.567 & 早稲田大学 & 0.317 & 神学 & 0.612\\
日本女子大学 & 0.562 & 中央大学 & 0.317 & キリスト & 0.606\\ \hline
\end{tabular}
\label{table:wvf}
\end{table}


\subsection{WV\_ G}
単語ベクトルの次元数が50次元,反復回数が100,windowサイズを1000で生成したモデルの結果を\ref{table:wvg}に示す.

青山学院大学に近い単語は九州大学や東北大学,名古屋大学など立地的に遠い大学が出現した.
また $ 青山学院大学 - キリスト教 $ では,ミッション系の大学である同志社大学が得られたため,モデルの有効性は低いと考えられる.

\begin{table}[H]
\caption{WV\_ Gの検証結果}
\centering
\footnotesize
\begin{tabular}{ll|ll|ll}
\hline
\multicolumn{2}{c}{青山学院大学に近い大学} & \multicolumn{2}{c}{青山学院大学 - キリスト教} & \multicolumn{2}{c}{青山学院大学と明治学院大学で共通の近い単語}
% \multicolumn{2}{c}{} & \\ \hline
\\ \hline
大学名 & 類似度 & 大学名 & 類似度 & 単語 & 類似度の合計
\\ \hline \hline
九州大学 & 0.708 & 九州大学 & 0.556 & イギリス & 1.07\\
東北大学 & 0.52 & 名古屋大学 & 0.424 & 前期 & 1.005\\
明治学院大学 & 0.503 & 東北大学 & 0.392 & キリスト教 & 0.902\\
上智大学 & 0.451 & 東京大学 & 0.315 & 基本 & 0.883\\
名古屋大学 & 0.448 & 大阪大学 & 0.299 & 芸術 & 0.834\\
短期大学 & 0.442 & 同志社大学 & 0.289 & 神学 & 0.824\\
東京大学 & 0.44 & 立命館大学 & 0.256 & 併設 & 0.818\\
明治大学 & 0.42 & 中央大学 & 0.246 & キリスト & 0.808\\
立教大学 & 0.41 & 明治大学 & 0.232 & 教会 & 0.774\\
東京農業大学 & 0.41 & 日本女子大学 & 0.231 & 山手 & 0.688\\ \hline
\end{tabular}
\label{table:wvg}
\end{table}


\subsection{WV\_ H}
単語ベクトルの次元数が100次元,反復回数が100,windowサイズを1000で生成したモデルの結果を\ref{table:wvh}に示す.

青山学院大学に近い大学で,偏差値の近い明治大学や中央大学が得られた.
また,同じミッション系の大学として,明治学院大学や立教大学,同志社大学,上智大学も得られた.
一方で,$ 青山学院大学 - キリスト教 $ でミッション系の大学の同志社大学や,オックスフォード大学などの大学が得られた.

青山学院大学と明治学院大学で共通の近い単語では,ミッション系の大学に関する単語や神学という単語とともに,新しく統合という単語が得られた.
この単語は神学部や専門部の統合という歴史的な背景から得られたと考えられる.

\begin{table}[H]
\caption{WV\_ Hの検証結果}
\centering
\footnotesize
\begin{tabular}{ll|ll|ll}
\hline
\multicolumn{2}{c}{青山学院大学に近い大学} & \multicolumn{2}{c}{青山学院大学 - キリスト教} & \multicolumn{2}{c}{青山学院大学と明治学院大学で共通の近い単語}
% \multicolumn{2}{c}{} & \\ \hline
\\ \hline
大学名 & 類似度 & 大学名 & 類似度 & 単語 & 類似度の合計
\\ \hline \hline
明治大学 & 0.494 & 明治大学 & 0.263 & キリスト教 & 0.84\\
明治学院大学 & 0.457 & 北里大学 & 0.255 & イギリス & 0.764\\
立教大学 & 0.435 & 東京外国語大学 & 0.23 & キリスト & 0.703\\
同志社大学 & 0.429 & 同志社大学 & 0.225 & 教会 & 0.671\\
北里大学 & 0.428 & 学習院大学 & 0.219 & 前期 & 0.661\\
東京外国語大学 & 0.424 & オックスフォード大学 & 0.197 & 神学 & 0.655\\
関西学院大学 & 0.377 & 東北大学 & 0.189 & 統合 & 0.613\\
短期大学 & 0.375 & 名古屋大学 & 0.185 & 併設 & 0.608\\
上智大学 & 0.344 & 九州大学 & 0.175 & 山手 & 0.606\\
中央大学 & 0.337 & 関西学院大学 & 0.156 & チャペル & 0.605\\ \hline
\end{tabular}
\label{table:wvh}
\end{table}
\section{GloVeのモデル検証結果}
本節ではGloVeのモデル検証結果をまとめた.
GloVeで生成したモデル名と,対応するパラメータの一覧を表 \ref{table:gvResultAll}に示す.

\begin{table}[htbp]
\caption{GloVeパラメータの詳細}
\centering
\begin{tabular}{llll}
\hline
モデル名 & 単語ベクトルの次元数 & 反復回数 & windowサイズ
\\ \hline \hline
GV\_ A & 50 & 10 & 10\\ \hline
GV\_ B & 100 & 10 & 10\\ \hline
GV\_ C & 50 & 100 & 10\\ \hline
GV\_ D & 100 & 100 & 10\\ \hline
GV\_ E & 50 & 10 & 1000\\ \hline
GV\_ F & 100 & 10 & 1000\\ \hline
GV\_ G & 50 & 100 & 1000\\ \hline
GV\_ H & 100 & 100 & 1000\\ \hline
\end{tabular}
\label{table:gvResultAll}
\end{table}

\subsection{GV\_ A}
単語ベクトルの次元数が50次元,反復回数が10,windowサイズを10で生成したモデルの結果を\ref{table:gva}に示す.

このモデルからは,青山学院大学に近い大学として中央大学や立教大学,上智大学,明治大学,法政大学などの大学が得られた.
これらの大学は総合大学である点や,ミッション系大学である点などの類似点がある.
さらに,学習したデータでは考慮されなかった偏差値の近さも表現されている.

また,$ 青山学院大学 - キリスト教 $ では,上位10校からはミッション系の大学が出現しなかった.
しかし東京工業大学や大阪体育大学などは関連性が低いと考えられる.
国際連合大学は大学ではないが,大学院は存在し青山学院大学と立地は非常に近いため出現したと考えられる.

青山学院大学と明治学院大学で共通の近い単語に関しては,学習不足であった.

\begin{table}[H]
\caption{GV\_ Aの検証結果}
\centering
\footnotesize
\begin{tabular}{ll|ll|ll}
\hline
\multicolumn{2}{c}{青山学院大学に近い大学} & \multicolumn{2}{c}{青山学院大学 - キリスト教} & \multicolumn{2}{c}{青山学院大学と明治学院大学で共通の近い単語}
% \multicolumn{2}{c}{} & \\ \hline
\\ \hline
大学名 & 類似度 & 大学名 & 類似度 & 単語 & 類似度の合計
\\ \hline \hline
中央大学 & 0.764 & 中央大学 & 0.559 & 学位 & 0.922\\
立教大学 & 0.747 & 東京工芸大学 & 0.557 & & \\
上智大学 & 0.71 & 立教大学 & 0.537 & & \\
駒澤大学 & 0.71 & 法政大学 & 0.506 & & \\
明治大学 & 0.683 & 国際連合大学 & 0.499 & & \\
学習院大学 & 0.672 & 関西大学 & 0.482 & & \\
法政大学 & 0.671 & 明治大学 & 0.482 & & \\
実践女子大学 & 0.623 & 大阪芸術大学 & 0.472 & & \\
東京理科大学 & 0.616 & 大阪体育大学 & 0.471 & & \\
成蹊大学 & 0.609 & 金沢医科大学 & 0.461 & & \\ \hline
\end{tabular}
\label{table:gva}
\end{table}


\subsection{GV\_ B}
単語ベクトルの次元数が100次元,反復回数が10,windowサイズを10で生成したモデルの結果を\ref{table:gvb}に示す.

青山学院大学に近い大学として,Word2Vecの方では見られなかった愛知淑徳大学や大阪体育大学,芝浦工業大学などが得られた.
これらの大学が学部構成が大きく異なる上,ミッション系の大学ではないためモデルの精度としては妥当性が低いと言える.

$ 青山学院大学 - キリスト教 $ の結果から,中央大学を除いて関連性の低い大学が出現した.

このモデルでは学習不足により青山学院大学と明治学院大学で共通の近い単語が得られなかった.


\begin{table}[H]
\caption{GV\_ Bの検証結果}
\centering
\footnotesize
\begin{tabular}{ll|ll|ll}
\hline
\multicolumn{2}{c}{青山学院大学に近い大学} & \multicolumn{2}{c}{青山学院大学 - キリスト教} & \multicolumn{2}{c}{青山学院大学と明治学院大学で共通の近い単語}
% \multicolumn{2}{c}{} & \\ \hline
\\ \hline
大学名 & 類似度 & 大学名 & 類似度 & 単語 & 類似度の合計
\\ \hline \hline
中央大学 & 0.639 & 東京工芸大学 & 0.52 & & \\
立教大学 & 0.588 & 中央大学 & 0.471 & & \\
学習院大学 & 0.562 & 大阪体育大学 & 0.46 & & \\
大東文化大学 & 0.551 & 大東文化大学 & 0.449 & & \\
明治大学 & 0.545 & 大阪芸術大学 & 0.435 & & \\
上智大学 & 0.537 & 神戸市外国語大学 & 0.415 & & \\
愛知淑徳大学 & 0.532 & 金沢医科大学 & 0.412 & & \\
大阪体育大学 & 0.53 & 京都女子大学 & 0.408 & & \\
芝浦工業大学 & 0.513 & 名城大学 & 0.407 & & \\
駒澤大学 & 0.496 & 会津大学 & 0.396 & & \\ \hline
\end{tabular}
\label{table:gvb}
\end{table}

\subsection{GV\_ C}
単語ベクトルの次元数が50次元,反復回数が100,windowサイズを10で生成したモデルの結果を\ref{table:gvc}に示す.

% 青山学院大学に近い大学として得られた大学は偏差値,立地の観点から妥当なものと考えられる.
青山学院大学に近い大学として得られた大学は,学部構成などの観点から妥当性の高い結果となっていると考えられる.
しかし $ 青山学院大学 - キリスト教 $ から,上智大学と立教大学が得られた.
これらの大学はミッション系であるため,モデルの妥当性は低いと考えられる.
また,このモデルでは学習不足により青山学院大学と明治学院大学で共通の近い単語が得られなかった.

\begin{table}[H]
\caption{GV\_ Cの検証結果}
\centering
\footnotesize
\begin{tabular}{ll|ll|ll}
\hline
\multicolumn{2}{c}{青山学院大学に近い大学} & \multicolumn{2}{c}{青山学院大学 - キリスト教} & \multicolumn{2}{c}{青山学院大学と明治学院大学で共通の近い単語}
% \multicolumn{2}{c}{} & \\ \hline
\\ \hline
大学名 & 類似度 & 大学名 & 類似度 & 単語 & 類似度の合計
\\ \hline \hline
立教大学 & 0.691 & 中央大学 & 0.476 & & \\
中央大学 & 0.667 & 明治大学 & 0.466 & & \\
上智大学 & 0.632 & 成蹊大学 & 0.437 & & \\
明治大学 & 0.616 & 法政大学 & 0.426 & & \\
駒澤大学 & 0.612 & 武蔵大学 & 0.407 & & \\
法政大学 & 0.601 & 上智大学 & 0.404 & & \\
学習院大学 & 0.563 & 駒澤大学 & 0.391 & & \\
早稲田大学 & 0.497 & 立教大学 & 0.366 & & \\
成蹊大学 & 0.479 & 福岡大学 & 0.354 & & \\
東京理科大学 & 0.476 & 成城大学 & 0.347 & & \\ \hline
\end{tabular}
\label{table:gvc}
\end{table}

\subsection{GV\_ D}
単語ベクトルの次元数が100次元,反復回数が100,windowサイズを10で生成したモデルの結果を\ref{table:gvd}に示す.

% 青山学院大学に近い大学として新しく大東文化大学が得られた.
% しかし評価項目を考慮すると,法政大学や上智大学よりも類似度が高い結果は妥当性が低いと考えられる.
青山学院大学に近い大学として得られた大学は学部構成の観点から近いものであると考えられる.
しかし芝浦工業大学は単科大学であるため,学部構成の観点から遠いと言える.

$ 青山学院大学 - キリスト教 $ の結果から,淑徳大学が新しく得られた.
キリスト教の要素を除いた結果,仏教系の大学が出現した様に見えるが,学部構成的に類似しているとは言い難い.

また,このモデルでは学習不足により青山学院大学と明治学院大学で共通の近い単語が得られなかった.

\begin{table}[H]
\caption{GV\_ Dの検証結果}
\centering
\footnotesize
\begin{tabular}{ll|ll|ll}
\hline
\multicolumn{2}{c}{青山学院大学に近い大学} & \multicolumn{2}{c}{青山学院大学 - キリスト教} & \multicolumn{2}{c}{青山学院大学と明治学院大学で共通の近い単語}
% \multicolumn{2}{c}{} & \\ \hline
\\ \hline
大学名 & 類似度 & 大学名 & 類似度 & 単語 & 類似度の合計
\\ \hline \hline
中央大学 & 0.642 & 中央大学 & 0.55 & & \\
明治大学 & 0.549 & 法政大学 & 0.412 & & \\
立教大学 & 0.542 & 明治大学 & 0.377 & & \\
学習院大学 & 0.539 & 関西大学 & 0.375 & & \\
大東文化大学 & 0.52 & 淑徳大学 & 0.37 & & \\
上智大学 & 0.52 & 武蔵大学 & 0.359 & & \\
成城大学 & 0.509 & 東洋大学 & 0.355 & & \\
駒澤大学 & 0.486 & 東京理科大学 & 0.349 & & \\
法政大学 & 0.474 & 駒澤大学 & 0.348 & & \\
芝浦工業大学 & 0.455 & 東京工科大学 & 0.348 & & \\ \hline
\end{tabular}
\label{table:gvd}
\end{table}

\subsection{GV\_ E}
単語ベクトルの次元数が50次元,反復回数が10,windowサイズを1000で生成したモデルの結果を\ref{table:gve}に示す.

青山学院大学に近い大学からは,女子美術大学がもっとも類似度が高い大学として得られた.
学習したデータにキャンパスの所在地は考慮されていないが,相模原という単語に影響を受けた可能性がある.

$ 青山学院大学 - キリスト教 $ の結果からは相模女子大学が得られた.
こちらも左の結果と同様に相模原という単語に影響を受けた可能性がある.
また,このモデルの結果からは単科大学が出力される傾向が強かった.
そのため,モデルの精度としては低いと考えられる.

青山学院大学と明治学院大学で共通の近い単語は,ミッション系の大学を連想する単語が出現したものの,類似度の合計値が高い単語は関連性が不明確なものであった.

\begin{table}[H]
\caption{GV\_ Eの検証結果}
\centering
\footnotesize
\begin{tabular}{ll|ll|ll}
\hline
\multicolumn{2}{c}{青山学院大学に近い大学} & \multicolumn{2}{c}{青山学院大学 - キリスト教} & \multicolumn{2}{c}{青山学院大学と明治学院大学で共通の近い単語}
% \multicolumn{2}{c}{} & \\ \hline
\\ \hline
大学名 & 類似度 & 大学名 & 類似度 & 単語 & 類似度の合計
\\ \hline \hline
女子美術大学 & 0.647 & 相模女子大学 & 0.442 & 芸術 & 1.024\\
立教大学 & 0.631 & 神奈川工科大学 & 0.396 & コミュニティ & 0.93\\
聖心女子大学 & 0.622 & 芝浦工業大学 & 0.391 & 音楽 & 0.925\\
東京工業大学 & 0.615 & 北海道教育大学 & 0.389 & 学位 & 0.874\\
津田塾大学 & 0.613 & 女子美術大学 & 0.373 & 礼拝 & 0.854\\
上智大学 & 0.611 & 東京農業大学 & 0.372 & 心理 & 0.854\\
関西学院大学 & 0.61 & 国際連合大学 & 0.362 & 合同 & 0.825\\
明治学院大学 & 0.605 & 目白大学 & 0.362 & & \\
帝京大学 & 0.602 & 創価大学 & 0.359 & & \\
東京農業大学 & 0.599 & 帝京大学 & 0.351 & & \\ \hline
\end{tabular}
\label{table:gve}
\end{table}

\subsection{GV\_ F}
単語ベクトルの次元数が100次元,反復回数が10,windowサイズを1000で生成したモデルの結果を\ref{table:gvf}に示す.

青山学院大学に近い大学からは,関西学院大学や聖心女子大学,明治学院大学などのミッション系の大学が得られた.
聖心女子大学や東京工業大学,女子美術大学等は単科大学なので,学部構成の観点から類似度は高くないはずである.

$ 青山学院大学 - キリスト教 $ の結果からは,このモデルでも単科大学の大学が上位に表示された.

また,青山学院大学と明治学院大学で共通の近い単語からは,関係性のある単語が取得できなかった.

\begin{table}[H]
\caption{GV\_ Fの検証結果}
\centering
\footnotesize
\begin{tabular}{ll|ll|ll}
\hline
\multicolumn{2}{c}{青山学院大学に近い大学} & \multicolumn{2}{c}{青山学院大学 - キリスト教} & \multicolumn{2}{c}{青山学院大学と明治学院大学で共通の近い単語}
% \multicolumn{2}{c}{} & \\ \hline
\\ \hline
大学名 & 類似度 & 大学名 & 類似度 & 単語 & 類似度の合計
\\ \hline \hline
関西学院大学 & 0.541 & 神奈川工科大学 & 0.344 & 女学院 & 0.753\\
聖心女子大学 & 0.525 & 相模女子大学 & 0.334 & 英和 & 0.742\\
早稲田大学 & 0.514 & 北海道教育大学 & 0.333 & 芸術 & 0.731\\
明治学院大学 & 0.513 & 目白大学 & 0.315 & 前期 & 0.705\\
実践女子大学 & 0.503 & 国際連合大学 & 0.315 & 音楽 & 0.676\\
東京工業大学 & 0.495 & 関西学院大学 & 0.299 & コミュニティ & 0.674\\
女子美術大学 & 0.492 & 関東学院大学 & 0.293 & 貢献 & 0.631\\
津田塾大学 & 0.489 & 会津大学 & 0.285 & & \\
帝京大学 & 0.485 & 大妻女子大学 & 0.283 & & \\
東京大学 & 0.474 & 聖心女子大学 & 0.279 & & \\ \hline
\end{tabular}
\label{table:gvf}
\end{table}


\subsection{GV\_ G}
単語ベクトルの次元数が50次元,反復回数が100,windowサイズを1000で生成したモデルの結果を\ref{table:gvg}に示す.

このモデルでも青山学院大学に近い大学として関西学院大学が得られた.
しかし女子美術大学や国際大学,愛知淑徳大学などは関連性が見出せなかった.

$ 青山学院大学 - キリスト教 $ の結果からは,単科大学の出現が多かった.

また,青山学院大学と明治学院大学で共通の近い単語からは,関係性のある単語が取得できなかった.

\begin{table}[H]
\caption{GV\_ Gの検証結果}
\centering
\footnotesize
\begin{tabular}{ll|ll|ll}
\hline
\multicolumn{2}{c}{青山学院大学に近い大学} & \multicolumn{2}{c}{青山学院大学 - キリスト教} & \multicolumn{2}{c}{青山学院大学と明治学院大学で共通の近い単語}
% \multicolumn{2}{c}{} & \\ \hline
\\ \hline
大学名 & 類似度 & 大学名 & 類似度 & 単語 & 類似度の合計
\\ \hline \hline
関西学院大学 & 0.505 & 神奈川工科大学 & 0.41 & 女学院 & 0.63\\
女子美術大学 & 0.497 & 東京農業大学 & 0.38 & 芸術 & 0.612\\
聖心女子大学 & 0.47 & 明治大学 & 0.377 & 英和 & 0.572\\
立教大学 & 0.459 & 相模女子大学 & 0.372 & 学位 & 0.485\\
国際大学 & 0.456 & 東京情報大学 & 0.369 & & \\
相模女子大学 & 0.445 & 帝京大学 & 0.355 & & \\
同志社大学 & 0.445 & 神奈川大学 & 0.35 & & \\
津田塾大学 & 0.425 & 帝京平成大学 & 0.347 & & \\
愛知淑徳大学 & 0.421 & 東京農工大学 & 0.333 & & \\
東京情報大学 & 0.411 & 福島県立医科大学 & 0.328 & & \\ \hline
\end{tabular}
\label{table:gvg}
\end{table}

\subsection{GV\_ H}
単語ベクトルの次元数が100次元,反復回数が100,windowサイズを1000で生成したモデルの結果を\ref{table:gvh}に示す.

このモデルでは,青山学院大学にもっとも近い大学として,関西学院大学が得られた.
また青山学院女子短期大学やオックスフォード大学など,他のモデルではあまり見られなかった大学が得られた.

$ 青山学院大学 - キリスト教 $ の結果からは,もっとも近い大学として関西学院大学が得られたが,ミッション系の大学であるため,モデルの精度は妥当ではないと考えられる.

青山学院大学と明治学院大学で共通の近い単語からは,ミッション系の大学を連想する単語と,統合という歴史的な背景を連想する単語が得られた.

\begin{table}[H]
\caption{GV\_ Hの検証結果}
\centering
\footnotesize
\begin{tabular}{ll|ll|ll}
\hline
\multicolumn{2}{c}{青山学院大学に近い大学} & \multicolumn{2}{c}{青山学院大学 - キリスト教} & \multicolumn{2}{c}{青山学院大学と明治学院大学で共通の近い単語}
% \multicolumn{2}{c}{} & \\ \hline
\\ \hline
大学名 & 類似度 & 大学名 & 類似度 & 単語 & 類似度の合計
\\ \hline \hline
関西学院大学 & 0.353 & 関西学院大学 & 0.315 & 聖書 & 0.471\\
東京外国語大学 & 0.322 & 東京農業大学 & 0.272 & 定期 & 0.458\\
東京農業大学 & 0.271 & 神戸大学 & 0.214 & 統合 & 0.455\\
静岡大学 & 0.27 & 東京理科大学 & 0.213 & 女学院 & 0.451\\
青山学院女子短期大学 & 0.257 & 横浜市立大学 & 0.213 & 英和 & 0.426\\
オックスフォード大学 & 0.242 & 静岡大学 & 0.18 & チャペル & 0.369\\
神戸大学 & 0.235 & 昭和女子大学 & 0.179 & 前期 & 0.365\\
東京工業大学 & 0.229 & 北海道大学 & 0.179 & キリスト & 0.359\\
九州大学 & 0.229 & 明治大学 & 0.175 & キリスト教 & 0.357\\
筑波大学 & 0.221 & 九州大学 & 0.175 & 申し出 & 0.351\\ \hline
\end{tabular}
\label{table:gvh}
\end{table}

