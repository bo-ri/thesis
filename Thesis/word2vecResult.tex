\section{Word2Vecのモデル検証結果}
Word2Vecのモデル検証結果を示す.
今回検証したモデル名と,対応するパラメータの一覧を表 \ref{table:wvResultAll}に示す.

\begin{table}[htbp]
\caption{Word2Vecパラメータの詳細}
\centering
\begin{tabular}{llll}
\hline
モデル名 & 単語ベクトルの次元数 & 反復回数 & windowサイズ
\\ \hline \hline
WV\_ A & 50 & 10 & 10\\ \hline
WV\_ B & 100 & 10 & 10\\ \hline
WV\_ C & 50 & 100 & 10\\ \hline
WV\_ D & 100 & 100 & 10\\ \hline
WV\_ E & 50 & 10 & 1000\\ \hline
WV\_ F & 100 & 10 & 1000\\ \hline
WV\_ G & 50 & 100 & 1000\\ \hline
WV\_ H & 100 & 100 & 1000\\ \hline
\end{tabular}
\label{table:wvResultAll}
\end{table}

\subsection{WV\_ A}
単語ベクトルの次元数が50次元,反復回数とwindowサイズがそれぞれ10で生成したモデルの結果を表 \ref{table:wva}に示す.

青山学院大学に2番目に近い大学として明治学院大学が出現したが,それ以外の結果は関連性が不透明である.
この結果からは有効なモデルだと判断できない.

\begin{table}[H]
\caption{WV\_ Aの検証結果}
\centering
\footnotesize
% \scriptsize
\begin{tabular}{ll|ll|ll}
\hline
\multicolumn{2}{c}{青山学院大学に近い大学} & \multicolumn{2}{c}{青山学院大学 - キリスト教} & \multicolumn{2}{c}{青山学院大学と明治学院大学で共通の近い単語}
% \multicolumn{2}{c}{} & \\ \hline
\\ \hline
% 大学名 & 類似度 & 大学名 & 類似度 & 単語 & 青山学院大学との類似度 & 明治学院大学との類似度
大学名 & 類似度 & 大学名 & 類似度 & 単語 & 類似度の合計
\\ \hline \hline
聖心女子大学 & 0.74 & 目白大学 & 0.62 & 女学院 & 1.094\\
明治学院大学 & 0.722 & 帝京大学 & 0.583 & 英和 & 0.968\\
昭和女子大学 & 0.704 & 芝浦工業大学 & 0.552 & 芸術 & 0.742\\
帝京大学 & 0.703 & 女子美術大学 & 0.546 & ライン & 0.634\\
清泉女子大学 & 0.691 & 東京薬科大学 & 0.545 & 山手 & 0.579\\
実践女子大学 & 0.649 & 東京家政大学 & 0.544 & 校友 & 0.439\\
白百合女子大学 & 0.649 & 東京情報大学 & 0.542 & & \\
津田塾大学 & 0.638 & 東京電機大学 & 0.529 & & \\
東京女子体育大学 & 0.637 & 東京理科大学 & 0.525 & & \\
女子美術大学 & 0.636 & 多摩美術大学 & 0.524 & & \\ \hline
\end{tabular}
\label{table:wva}
\end{table}



\subsection{WV\_ B}
単語ベクトルの次元数が100次元,反復回数とwindowサイズがそれぞれ10で生成したモデルの結果を\ref{table:wvb}に示す.

青山学院大学に近い大学として,ミッション系の大学で偏差値も比較的近いと考えられる上智大学が得られた.
また $ 青山学院大学 - キリスト教 $ の結果から上位10校にはミッション系の大学が含まれなかった.
さらに,青山学院大学と明治学院大学で共通の近い意味の単語から,キリスト教などの単語が得られた.

しかし青山学院大学に近い大学の順位を全体的に見ると,関連性があると断言できない.
\begin{table}[H]
\caption{WV\_ Bの検証結果}
\centering
\footnotesize
\begin{tabular}{ll|ll|ll}
\hline
\multicolumn{2}{c}{青山学院大学に近い大学} & \multicolumn{2}{c}{青山学院大学 - キリスト教} & \multicolumn{2}{c}{青山学院大学と明治学院大学で共通の近い単語}
% \multicolumn{2}{c}{} & \\ \hline
\\ \hline
大学名 & 類似度 & 大学名 & 類似度 & 単語 & 類似度の合計
\\ \hline \hline
上智大学 & 0.372 & 成蹊大学 & 0.317 & 定期 & 0.531\\
東京外国語大学 & 0.308 & 駒澤大学 & 0.293 & 神学 & 0.45\\
駒澤大学 & 0.308 & 獨協大学 & 0.262 & 芸術 & 0.415\\
東京大学 & 0.295 & 東京理科大学 & 0.253 & 監督 & 0.37\\
筑波大学 & 0.294 & 中央大学 & 0.252 & キリスト & 0.364\\
早稲田大学 & 0.289 & 名古屋大学 & 0.221 & イギリス & 0.363\\
中央大学 & 0.276 & 筑波大学 & 0.204 & キリスト教 & 0.323\\
オックスフォード大学 & 0.271 & 明治大学 & 0.19 & 合同 & 0.304\\
東京農業大学 & 0.256 & 大阪大学 & 0.167 & 前期 & 0.291\\
成蹊大学 & 0.256 & 関西学院大学 & 0.167 & チャペル & 0.272\\ \hline
\end{tabular}
\label{table:wvb}
\end{table}


\subsection{WV\_ C}
単語ベクトルの次元数が50次元,反復回数が100,windowサイズを10で生成したモデルの結果を\ref{table:wvc}に示す.

青山学院大学に近い大学で上智大学や明治学院大学などが出現したが,単語ベクトルの類似度の高さでは清泉女子大学と聖心女子大学よりも低い.
どちらもミッション系の大学であるが,偏差値などの観点から上智大学や明治学院大学の方が近いと考えられる.

青山学院大学と明治学院大学で共通の近い単語では,神学という単語が出現した.
しかし,それぞれの大学との類似度の合計は他の単語に対して高くないため,学習不足であると考えられる.

\begin{table}[H]
\caption{WV\_ Cの検証結果}
\centering
\footnotesize
\begin{tabular}{ll|ll|ll}
\hline
\multicolumn{2}{c}{青山学院大学に近い大学} & \multicolumn{2}{c}{青山学院大学 - キリスト教} & \multicolumn{2}{c}{青山学院大学と明治学院大学で共通の近い単語}
% \multicolumn{2}{c}{} & \\ \hline
\\ \hline
大学名 & 類似度 & 大学名 & 類似度 & 単語 & 類似度の合計
\\ \hline \hline
清泉女子大学 & 0.677 & 東京薬科大学 & 0.557 & 女学院 & 0.981\\
聖心女子大学 & 0.674 & 帝京大学 & 0.539 & 英和 & 0.893\\
上智大学 & 0.633 & 中央大学 & 0.525 & 芸術 & 0.819\\
明治学院大学 & 0.616 & 目白大学 & 0.516 & ライン & 0.625\\
帝京大学 & 0.599 & 東京情報大学 & 0.498 & 学位 & 0.538\\
大東文化大学 & 0.585 & 女子美術大学 & 0.472 & 山手 & 0.538\\
立教大学 & 0.58 & 桜美林大学 & 0.469 & 校友 & 0.515\\
実践女子大学 & 0.578 & 東京電機大学 & 0.46 & 神学 & 0.463\\
桜美林大学 & 0.563 & 成蹊大学 & 0.458 & & \\
昭和女子大学 & 0.561 & 工学院大学 & 0.45 & & \\ \hline
\end{tabular}
\label{table:wvc}
\end{table}


\subsection{WV\_ D}
単語ベクトルの次元数が100次元,反復回数が100,windowサイズを10で生成したモデルの結果を\ref{table:wvd}に示す.

青山学院大学に近い大学として,上智大学や中央大学,学習院大学等が得られた.
しかし類似度が高い大学で仏教系大学である駒澤大学などが出現した.
青山学院大学と明治学院大学で共通の近い単語はキリストや礼拝,教会などミッション系の大学を連想させるような単語が新しく出現した.


\begin{table}[H]
\caption{WV\_ Dの検証結果}
\centering
\footnotesize
\begin{tabular}{ll|ll|ll}
\hline
\multicolumn{2}{c}{青山学院大学に近い大学} & \multicolumn{2}{c}{青山学院大学 - キリスト教} & \multicolumn{2}{c}{青山学院大学と明治学院大学で共通の近い単語}
% \multicolumn{2}{c}{} & \\ \hline
\\ \hline
大学名 & 類似度 & 大学名 & 類似度 & 単語 & 類似度の合計
\\ \hline \hline
上智大学 & 0.393 & 駒澤大学 & 0.337 & キリスト & 0.493\\
駒澤大学 & 0.364 & 中央大学 & 0.314 & 定期 & 0.491\\
東京外国語大学 & 0.304 & 獨協大学 & 0.261 & キリスト教 & 0.441\\
中央大学 & 0.3 & 関西学院大学 & 0.244 & 神学 & 0.433\\
東京農業大学 & 0.288 & 東洋大学 & 0.224 & 礼拝 & 0.424\\
関西学院大学 & 0.283 & 成蹊大学 & 0.208 & 宗教 & 0.416\\
早稲田大学 & 0.254 & 名古屋大学 & 0.178 & 芸術 & 0.39\\
筑波大学 & 0.253 & 明治大学 & 0.177 & 教会 & 0.358\\
獨協大学 & 0.253 & 筑波大学 & 0.166 & 基本 & 0.34\\
学習院大学 & 0.249 & 法政大学 & 0.165 & チャペル & 0.311\\ \hline
\end{tabular}
\label{table:wvd}
\end{table}

\subsection{WV\_ E}
単語ベクトルの次元数が50次元,反復回数が10,windowサイズを1000で生成したモデルの結果を\ref{table:wve}に示す.

このモデルでは,青山学院大学に近い大学として中央大学,法政大学,立教大学,明治大学が類似度の高い大学としてあげられた.
また,$ 青山学院大学 - キリスト教 $ はミッション系の大学が出現せずに法政大学や明治大学を得ることができた.

しかし,青山学院大学と明治学院大学で共通の近い単語から,神学やミッション系の大学に関する単語が出現しなくなった.

\begin{table}[H]
\caption{WV\_ Eの検証結果}
\centering
\footnotesize
\begin{tabular}{ll|ll|ll}
\hline
\multicolumn{2}{c}{青山学院大学に近い大学} & \multicolumn{2}{c}{青山学院大学 - キリスト教} & \multicolumn{2}{c}{青山学院大学と明治学院大学で共通の近い単語}
% \multicolumn{2}{c}{} & \\ \hline
\\ \hline
大学名 & 類似度 & 大学名 & 類似度 & 単語 & 類似度の合計
\\ \hline \hline
東洋大学 & 0.69 & 法政大学 & 0.558 & 芸術 & 0.787\\
日本女子大学 & 0.651 & 明治大学 & 0.527 & 音楽 & 0.741\\
聖心女子大学 & 0.636 & 立命館大学 & 0.49 & イギリス & 0.732\\
立命館大学 & 0.623 & 東洋大学 & 0.482 & 女学院 & 0.714\\
昭和女子大学 & 0.623 & 昭和女子大学 & 0.465 & バス & 0.711\\
中央大学 & 0.607 & 千葉工業大学 & 0.463 & コミュニティ & 0.643\\
法政大学 & 0.601 & 横浜市立大学 & 0.459 & 相談 & 0.601\\
立教大学 & 0.595 & 中央大学 & 0.455 & & \\
明治大学 & 0.584 & 金沢工業大学 & 0.455 & & \\
東京電機大学 & 0.569 & 神奈川大学 & 0.433 & & \\ \hline
\end{tabular}
\label{table:wve}
\end{table}

\subsection{WV\_ F}
単語ベクトルの次元数が100次元,反復回数が10,windowサイズを1000で生成したモデルの結果を\ref{table:wvf}に示す.

青山学院大学に近い大学は近い偏差値の大学やミッション系の大学が多く得られた.
青山学院大学と明治学院大学で共通の近い単語から,商業という単語が新しく得られた.
これは1944年に専門部を閉鎖し,明治学院に合同した際の高等商業学部から関連性があると考えられる.

\begin{table}[H]
\caption{WV\_ Fの検証結果}
\centering
\footnotesize
\begin{tabular}{ll|ll|ll}
\hline
\multicolumn{2}{c}{青山学院大学に近い大学} & \multicolumn{2}{c}{青山学院大学 - キリスト教} & \multicolumn{2}{c}{青山学院大学と明治学院大学で共通の近い単語}
% \multicolumn{2}{c}{} & \\ \hline
\\ \hline
大学名 & 類似度 & 大学名 & 類似度 & 単語 & 類似度の合計
\\ \hline \hline
明治大学 & 0.729 & 明治大学 & 0.426 & 心理 & 0.857\\
北里大学 & 0.713 & 法政大学 & 0.423 & 併設 & 0.85\\
法政大学 & 0.707 & 北里大学 & 0.385 & 統合 & 0.843\\
明治学院大学 & 0.66 & 九州大学 & 0.362 & 商業 & 0.817\\
短期大学 & 0.658 & 立命館大学 & 0.341 & 前期 & 0.794\\
上智大学 & 0.654 & 名古屋大学 & 0.337 & キリスト教 & 0.791\\
立教大学 & 0.65 & 駒澤大学 & 0.331 & イギリス & 0.72\\
中央大学 & 0.6 & 大阪大学 & 0.321 & 教会 & 0.679\\
早稲田大学 & 0.567 & 早稲田大学 & 0.317 & 神学 & 0.612\\
日本女子大学 & 0.562 & 中央大学 & 0.317 & キリスト & 0.606\\ \hline
\end{tabular}
\label{table:wvf}
\end{table}


\subsection{WV\_ G}
単語ベクトルの次元数が50次元,反復回数が100,windowサイズを1000で生成したモデルの結果を\ref{table:wvg}に示す.

青山学院大学に近い単語は九州大学や東北大学,名古屋大学など立地的に遠い大学が出現した.
また $ 青山学院大学 - キリスト教 $ では,ミッション系の大学である同志社大学が得られたため,モデルの有効性は低いと考えられる.

\begin{table}[H]
\caption{WV\_ Gの検証結果}
\centering
\footnotesize
\begin{tabular}{ll|ll|ll}
\hline
\multicolumn{2}{c}{青山学院大学に近い大学} & \multicolumn{2}{c}{青山学院大学 - キリスト教} & \multicolumn{2}{c}{青山学院大学と明治学院大学で共通の近い単語}
% \multicolumn{2}{c}{} & \\ \hline
\\ \hline
大学名 & 類似度 & 大学名 & 類似度 & 単語 & 類似度の合計
\\ \hline \hline
九州大学 & 0.708 & 九州大学 & 0.556 & イギリス & 1.07\\
東北大学 & 0.52 & 名古屋大学 & 0.424 & 前期 & 1.005\\
明治学院大学 & 0.503 & 東北大学 & 0.392 & キリスト教 & 0.902\\
上智大学 & 0.451 & 東京大学 & 0.315 & 基本 & 0.883\\
名古屋大学 & 0.448 & 大阪大学 & 0.299 & 芸術 & 0.834\\
短期大学 & 0.442 & 同志社大学 & 0.289 & 神学 & 0.824\\
東京大学 & 0.44 & 立命館大学 & 0.256 & 併設 & 0.818\\
明治大学 & 0.42 & 中央大学 & 0.246 & キリスト & 0.808\\
立教大学 & 0.41 & 明治大学 & 0.232 & 教会 & 0.774\\
東京農業大学 & 0.41 & 日本女子大学 & 0.231 & 山手 & 0.688\\ \hline
\end{tabular}
\label{table:wvg}
\end{table}


\subsection{WV\_ H}
単語ベクトルの次元数が100次元,反復回数が100,windowサイズを1000で生成したモデルの結果を\ref{table:wvh}に示す.

青山学院大学に近い大学で,偏差値の近い明治大学や中央大学が得られた.
また,同じミッション系の大学として,明治学院大学や立教大学,同志社大学,上智大学も得られた.
一方で,$ 青山学院大学 - キリスト教 $ でミッション系の大学の同志社大学や,オックスフォード大学などの大学が得られた.

青山学院大学と明治学院大学で共通の近い単語では,ミッション系の大学に関する単語や神学という単語とともに,新しく統合という単語が得られた.
この単語は神学部や専門部の統合という歴史的な背景から得られたと考えられる.

\begin{table}[H]
\caption{WV\_ Hの検証結果}
\centering
\footnotesize
\begin{tabular}{ll|ll|ll}
\hline
\multicolumn{2}{c}{青山学院大学に近い大学} & \multicolumn{2}{c}{青山学院大学 - キリスト教} & \multicolumn{2}{c}{青山学院大学と明治学院大学で共通の近い単語}
% \multicolumn{2}{c}{} & \\ \hline
\\ \hline
大学名 & 類似度 & 大学名 & 類似度 & 単語 & 類似度の合計
\\ \hline \hline
明治大学 & 0.494 & 明治大学 & 0.263 & キリスト教 & 0.84\\
明治学院大学 & 0.457 & 北里大学 & 0.255 & イギリス & 0.764\\
立教大学 & 0.435 & 東京外国語大学 & 0.23 & キリスト & 0.703\\
同志社大学 & 0.429 & 同志社大学 & 0.225 & 教会 & 0.671\\
北里大学 & 0.428 & 学習院大学 & 0.219 & 前期 & 0.661\\
東京外国語大学 & 0.424 & オックスフォード大学 & 0.197 & 神学 & 0.655\\
関西学院大学 & 0.377 & 東北大学 & 0.189 & 統合 & 0.613\\
短期大学 & 0.375 & 名古屋大学 & 0.185 & 併設 & 0.608\\
上智大学 & 0.344 & 九州大学 & 0.175 & 山手 & 0.606\\
中央大学 & 0.337 & 関西学院大学 & 0.156 & チャペル & 0.605\\ \hline
\end{tabular}
\label{table:wvh}
\end{table}