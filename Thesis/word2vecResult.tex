\section{Word2Vecのモデル検証結果}
Word2Vecのモデル検証結果を示す.
今回検証したモデル名と,対応するパラメータの一覧を表 \ref{table:wvResultAll}に示す.

\begin{table}[H]
\caption{Word2Vecパラメータの詳細}
\centering
\begin{tabular}{llll}
\hline
モデル名 & 単語ベクトルの次元数 & 反復回数 & windowサイズ
\\ \hline \hline
WV\_ A & 50 & 10 & 10\\ \hline
WV\_ B & 100 & 10 & 10\\ \hline
WV\_ C & 50 & 100 & 10\\ \hline
WV\_ D & 100 & 100 & 10\\ \hline
WV\_ E & 50 & 10 & 1000\\ \hline
WV\_ F & 100 & 10 & 1000\\ \hline
WV\_ G & 50 & 100 & 1000\\ \hline
WV\_ H & 100 & 100 & 1000\\ \hline
\end{tabular}
\label{table:wvResultAll}
\end{table}

\subsection{WV\_ A}
単語ベクトルの次元数が50次元,反復回数とwindowサイズがそれぞれ10で生成したモデルの結果を表 \ref{table:wva}に示す.

% 青山学院大学に2番目に近い大学として明治学院大学が出現したが,それ以外の結果は関連性が不透明である.
青山学院大学に近い大学として得られた大学のうち,ミッション系の大学は聖心女子大学や明治学院大学,清泉女子大学,白百合女子大学が挙げられる.
また,学部構成が近いのは明治学院大学や帝京大学が挙げられる.
$ 青山学院大学 - キリスト教 $ の結果からは,東京から始まる大学名が頻出するため,学習不足であると考えられる.
青山学院大学と明治学院大学で共通の近い単語の出力結果に関しては,明確にお互いの大学に共通する単語が出現していない.

\begin{table}[H]
\caption{WV\_ Aの検証結果}
\centering
\footnotesize
% \scriptsize
\begin{tabular}{ll|ll|ll}
\hline
\multicolumn{2}{c}{青山学院大学に近い大学} & \multicolumn{2}{c}{青山学院大学 - キリスト教} & \multicolumn{2}{c}{青山学院大学と明治学院大学で共通の近い単語}
% \multicolumn{2}{c}{} & \\ \hline
\\ \hline
% 大学名 & 類似度 & 大学名 & 類似度 & 単語 & 青山学院大学との類似度 & 明治学院大学との類似度
大学名 & 類似度 & 大学名 & 類似度 & 単語 & 類似度の合計
\\ \hline \hline
聖心女子大学 & 0.740 & 目白大学 & 0.620 & 女学院 & 1.094\\
明治学院大学 & 0.722 & 帝京大学 & 0.583 & 英和 & 0.968\\
昭和女子大学 & 0.704 & 芝浦工業大学 & 0.552 & 芸術 & 0.742\\
帝京大学 & 0.703 & 女子美術大学 & 0.546 & ライン & 0.634\\
清泉女子大学 & 0.691 & 東京薬科大学 & 0.545 & 山手 & 0.579\\
実践女子大学 & 0.649 & 東京家政大学 & 0.544 & 校友 & 0.439\\
白百合女子大学 & 0.649 & 東京情報大学 & 0.542 & & \\
津田塾大学 & 0.638 & 東京電機大学 & 0.529 & & \\
東京女子体育大学 & 0.637 & 東京理科大学 & 0.525 & & \\
女子美術大学 & 0.636 & 多摩美術大学 & 0.524 & & \\ \hline
\end{tabular}
\label{table:wva}
\end{table}



\subsection{WV\_ B}
単語ベクトルの次元数が100次元,反復回数とwindowサイズがそれぞれ10で生成したモデルの結果を表 \ref{table:wvb}に示す.

青山学院大学に近い大学として,ミッション系の大学で偏差値も比較的近いと考えられる上智大学が得られた.
% 2番目の東京外国語大学に関しては単科大学なので,評価基準の観点からは関係性は薄いと考える.
東京外国語大学は国際系の学部などが,青山学院大学と比較的近い存在であると考えた.
他には総合大学としては駒澤大学,東京大学,筑波大学,早稲田大学,中央大学等が挙げられる.

また $ 青山学院大学 - キリスト教 $ の結果から関西学院大学はミッション系の大学に該当する.
それ以外の大学であれば,獨協大学は青山学院大学と比較して総合大学として考えるならば学部数が少ない.
また東京理科大学は学部構成の観点から単科大学であり,駒澤大学は教育内容の観点から仏教であるため,関係性は薄いと考える.
さらに,青山学院大学と明治学院大学で共通の近い意味の単語から,キリスト教,神学,定期,合同などのミッション系の大学や歴史的背景を連想させるような単語が得られた.

\begin{table}[H]
\caption{WV\_ Bの検証結果}
\centering
\footnotesize
\begin{tabular}{ll|ll|ll}
\hline
\multicolumn{2}{c}{青山学院大学に近い大学} & \multicolumn{2}{c}{青山学院大学 - キリスト教} & \multicolumn{2}{c}{青山学院大学と明治学院大学で共通の近い単語}
% \multicolumn{2}{c}{} & \\ \hline
\\ \hline
大学名 & 類似度 & 大学名 & 類似度 & 単語 & 類似度の合計
\\ \hline \hline
上智大学 & 0.372 & 成蹊大学 & 0.317 & 定期 & 0.531\\
東京外国語大学 & 0.308 & 駒澤大学 & 0.293 & 神学 & 0.450\\
駒澤大学 & 0.308 & 獨協大学 & 0.262 & 芸術 & 0.415\\
東京大学 & 0.295 & 東京理科大学 & 0.253 & 監督 & 0.370\\
筑波大学 & 0.294 & 中央大学 & 0.252 & キリスト & 0.364\\
早稲田大学 & 0.289 & 名古屋大学 & 0.221 & イギリス & 0.363\\
中央大学 & 0.276 & 筑波大学 & 0.204 & キリスト教 & 0.323\\
オックスフォード大学 & 0.271 & 明治大学 & 0.190 & 合同 & 0.304\\
東京農業大学 & 0.256 & 大阪大学 & 0.167 & 前期 & 0.291\\
成蹊大学 & 0.256 & 関西学院大学 & 0.167 & チャペル & 0.272\\ \hline
\end{tabular}
\label{table:wvb}
\end{table}


\subsection{WV\_ C}
単語ベクトルの次元数が50次元,反復回数が100,windowサイズを10で生成したモデルの結果を表 \ref{table:wvc}に示す.

% 青山学院大学に近い大学で上智大学や明治学院大学などが出現したが,単語ベクトルの類似度の高さでは清泉女子大学と聖心女子大学よりも低い.
% どちらもミッション系の大学であるが,偏差値などの観点から上智大学や明治学院大学の方が近いと考えられる.
青山学院大学に近い大学で,上位の4校がミッション系の大学となった.
また全体的に女子大が多い出力となった.

$ 青山学院大学 - キリスト教 $ に関しては,最も近い大学に東京薬科大学が得られた.
また,東京電機大学や工学院大学,東京情報大学などの単科大学も同時に得られた.
学部構成を考えると近いとは言い難い結果となる.

青山学院大学と明治学院大学で共通の近い単語では,神学という単語が出現した.
しかし,それぞれの大学との類似度の合計は他の単語に対して高くないため,学習不足であると考えられる.

\begin{table}[H]
\caption{WV\_ Cの検証結果}
\centering
\footnotesize
\begin{tabular}{ll|ll|ll}
\hline
\multicolumn{2}{c}{青山学院大学に近い大学} & \multicolumn{2}{c}{青山学院大学 - キリスト教} & \multicolumn{2}{c}{青山学院大学と明治学院大学で共通の近い単語}
% \multicolumn{2}{c}{} & \\ \hline
\\ \hline
大学名 & 類似度 & 大学名 & 類似度 & 単語 & 類似度の合計
\\ \hline \hline
清泉女子大学 & 0.677 & 東京薬科大学 & 0.557 & 女学院 & 0.981\\
聖心女子大学 & 0.674 & 帝京大学 & 0.539 & 英和 & 0.893\\
上智大学 & 0.633 & 中央大学 & 0.525 & 芸術 & 0.819\\
明治学院大学 & 0.616 & 目白大学 & 0.516 & ライン & 0.625\\
帝京大学 & 0.599 & 東京情報大学 & 0.498 & 学位 & 0.538\\
大東文化大学 & 0.585 & 女子美術大学 & 0.472 & 山手 & 0.538\\
立教大学 & 0.580 & 桜美林大学 & 0.469 & 校友 & 0.515\\
実践女子大学 & 0.578 & 東京電機大学 & 0.460 & 神学 & 0.463\\
桜美林大学 & 0.563 & 成蹊大学 & 0.458 & & \\
昭和女子大学 & 0.561 & 工学院大学 & 0.450 & & \\ \hline
\end{tabular}
\label{table:wvc}
\end{table}


\subsection{WV\_ D}
単語ベクトルの次元数が100次元,反復回数が100,windowサイズを10で生成したモデルの結果を表 \ref{table:wvd}に示す.

% 青山学院大学に近い大学として,上智大学や中央大学,学習院大学等が得られた.
% しかし類似度が高い大学で仏教系大学である駒澤大学などが出現した.
青山学院大学に近い大学として,上智大学や中央大,関西学院大学,早稲田大学,学習院大学などが得られた.
これらの大学はミッション系や,総合大学であると言った共通点がある.
しかし,仏教系の大学である駒澤大学や,単科大学である東京農業大学が高い類似度で出現した.

$ 青山学院大学 - キリスト教 $ の結果から,関西学院大学がミッション系の大学として出力されてしまった.

青山学院大学と明治学院大学で共通の近い単語はキリストや礼拝,教会などミッション系の大学を連想させるような単語が得られた.
また,定期,神学といった歴史的背景を連想させる単語も得られた.


\begin{table}[H]
\caption{WV\_ Dの検証結果}
\centering
\footnotesize
\begin{tabular}{ll|ll|ll}
\hline
\multicolumn{2}{c}{青山学院大学に近い大学} & \multicolumn{2}{c}{青山学院大学 - キリスト教} & \multicolumn{2}{c}{青山学院大学と明治学院大学で共通の近い単語}
% \multicolumn{2}{c}{} & \\ \hline
\\ \hline
大学名 & 類似度 & 大学名 & 類似度 & 単語 & 類似度の合計
\\ \hline \hline
上智大学 & 0.393 & 駒澤大学 & 0.337 & キリスト & 0.493\\
駒澤大学 & 0.364 & 中央大学 & 0.314 & 定期 & 0.491\\
東京外国語大学 & 0.304 & 獨協大学 & 0.261 & キリスト教 & 0.441\\
中央大学 & 0.300 & 関西学院大学 & 0.244 & 神学 & 0.433\\
東京農業大学 & 0.288 & 東洋大学 & 0.224 & 礼拝 & 0.424\\
関西学院大学 & 0.283 & 成蹊大学 & 0.208 & 宗教 & 0.416\\
早稲田大学 & 0.254 & 名古屋大学 & 0.178 & 芸術 & 0.390\\
筑波大学 & 0.253 & 明治大学 & 0.177 & 教会 & 0.358\\
獨協大学 & 0.253 & 筑波大学 & 0.166 & 基本 & 0.340\\
学習院大学 & 0.249 & 法政大学 & 0.165 & チャペル & 0.311\\ \hline
\end{tabular}
\label{table:wvd}
\end{table}

\subsection{WV\_ E}
単語ベクトルの次元数が50次元,反復回数が10,windowサイズを1000で生成したモデルの結果を表 \ref{table:wve}に示す.

青山学院大学に近い大学として得られた東洋大学や立命館大学,中央大学や法政大学などは学部構成から意味の近さが説明できる.
% このモデルでは,青山学院大学に近い大学として中央大学,法政大学,立教大学,明治大学が類似度の高い大学としてあげられた.
% しかし,偏差値や立地を考えると類似度の値が相対的に低い.
また,$ 青山学院大学 - キリスト教 $ はミッション系の大学が出現せずに法政大学や明治大学,立命館大学,東洋大学が出力された.
これらは学部構成が似ているうえ,データでは考慮されていない偏差値も近い結果となっている.

青山学院大学と明治学院大学で共通の近い単語から,神学やミッション系の大学に関する単語が出現しなくなった.
ここから得られる単語では両校の関係性を証明できない.

\begin{table}[H]
\caption{WV\_ Eの検証結果}
\centering
\footnotesize
\begin{tabular}{ll|ll|ll}
\hline
\multicolumn{2}{c}{青山学院大学に近い大学} & \multicolumn{2}{c}{青山学院大学 - キリスト教} & \multicolumn{2}{c}{青山学院大学と明治学院大学で共通の近い単語}
% \multicolumn{2}{c}{} & \\ \hline
\\ \hline
大学名 & 類似度 & 大学名 & 類似度 & 単語 & 類似度の合計
\\ \hline \hline
東洋大学 & 0.690 & 法政大学 & 0.558 & 芸術 & 0.787\\
日本女子大学 & 0.651 & 明治大学 & 0.527 & 音楽 & 0.741\\
聖心女子大学 & 0.636 & 立命館大学 & 0.490 & イギリス & 0.732\\
立命館大学 & 0.623 & 東洋大学 & 0.482 & 女学院 & 0.714\\
昭和女子大学 & 0.623 & 昭和女子大学 & 0.465 & バス & 0.711\\
中央大学 & 0.607 & 千葉工業大学 & 0.463 & コミュニティ & 0.643\\
法政大学 & 0.601 & 横浜市立大学 & 0.459 & 相談 & 0.601\\
立教大学 & 0.595 & 中央大学 & 0.455 & & \\
明治大学 & 0.584 & 金沢工業大学 & 0.455 & & \\
東京電機大学 & 0.569 & 神奈川大学 & 0.433 & & \\ \hline
\end{tabular}
\label{table:wve}
\end{table}

\subsection{WV\_ F}
単語ベクトルの次元数が100次元,反復回数が10,windowサイズを1000で生成したモデルの結果を表 \ref{table:wvf}に示す.

% 青山学院大学に近い大学は近い偏差値の大学やミッション系の大学が多く得られた.
% このモデルでは北里大学の類似度が立教大学や上智大学などと比べて高く出ている.
青山学院大学に近い大学で,北里大学が得られたが,生命科学に力を入れているため,大学全体としては類似度が高いとは考えられない.
しかし青山学院大学にも理工学部生命科学コースが存在するため,その様なデータが反映されている可能性もある.

$ 青山学院大学 - キリスト教 $ では,青山学院大学に近い大学の結果からキリスト教に関する大学が消えた様な結果となった.
また,キリスト教という単語が消えた分,駒澤大学や立命館大学などがより類似度の高い結果となっている.

青山学院大学と明治学院大学で共通の近い単語から,商業という単語が新しく得られた.
これは1944年に専門部を閉鎖し,明治学院に合同した際の高等商業学部から関連性があると考えられる.

\begin{table}[H]
\caption{WV\_ Fの検証結果}
\centering
\footnotesize
\begin{tabular}{ll|ll|ll}
\hline
\multicolumn{2}{c}{青山学院大学に近い大学} & \multicolumn{2}{c}{青山学院大学 - キリスト教} & \multicolumn{2}{c}{青山学院大学と明治学院大学で共通の近い単語}
% \multicolumn{2}{c}{} & \\ \hline
\\ \hline
大学名 & 類似度 & 大学名 & 類似度 & 単語 & 類似度の合計
\\ \hline \hline
明治大学 & 0.729 & 明治大学 & 0.426 & 心理 & 0.857\\
北里大学 & 0.713 & 法政大学 & 0.423 & 併設 & 0.850\\
法政大学 & 0.707 & 北里大学 & 0.385 & 統合 & 0.843\\
明治学院大学 & 0.660 & 九州大学 & 0.362 & 商業 & 0.817\\
上智大学 & 0.654 & 立命館大学 & 0.341 & 前期 & 0.794\\
立教大学 & 0.650 & 名古屋大学 & 0.337 & キリスト教 & 0.791\\
中央大学 & 0.600 & 駒澤大学 & 0.331 & イギリス & 0.720\\
早稲田大学 & 0.567 & 大阪大学 & 0.321 & 教会 & 0.679\\
日本女子大学 & 0.562 & 早稲田大学 & 0.317 & 神学 & 0.612\\
専修大学 & 0.536 & 中央大学 & 0.317 & キリスト & 0.606\\ \hline
\end{tabular}
\label{table:wvf}
\end{table}


\subsection{WV\_ G}
単語ベクトルの次元数が50次元,反復回数が100,windowサイズを1000で生成したモデルの結果を表 \ref{table:wvg}に示す.

青山学院大学に近い大学の結果からは,このモデルでは国立大学が多く出力された.
出力された大学は主に総合大学で,明治学院大学や上智大学などのミッション系大学も得られた.

% 青山学院大学に近い単語は九州大学や東北大学,名古屋大学など立地的に遠い大学が出現した.
% 更に国立大学が多く見られる.
また $ 青山学院大学 - キリスト教 $ では,ミッション系の大学である同志社大学が得られたため,モデルの有効性は低いと考えられる.

青山学院大学と明治学院大学で共通の近い単語からは,ミッション系の単語などが得られた.
\begin{table}[H]
\caption{WV\_ Gの検証結果}
\centering
\footnotesize
\begin{tabular}{ll|ll|ll}
\hline
\multicolumn{2}{c}{青山学院大学に近い大学} & \multicolumn{2}{c}{青山学院大学 - キリスト教} & \multicolumn{2}{c}{青山学院大学と明治学院大学で共通の近い単語}
% \multicolumn{2}{c}{} & \\ \hline
\\ \hline
大学名 & 類似度 & 大学名 & 類似度 & 単語 & 類似度の合計
\\ \hline \hline
九州大学 & 0.708 & 九州大学 & 0.556 & イギリス & 1.070\\
東北大学 & 0.520 & 名古屋大学 & 0.424 & 前期 & 1.005\\
明治学院大学 & 0.503 & 東北大学 & 0.392 & キリスト教 & 0.902\\
上智大学 & 0.451 & 東京大学 & 0.315 & 基本 & 0.883\\
名古屋大学 & 0.448 & 大阪大学 & 0.299 & 芸術 & 0.834\\
東京大学 & 0.440 & 同志社大学 & 0.289 & 神学 & 0.824\\
明治大学 & 0.420 & 立命館大学 & 0.256 & 併設 & 0.818\\
立教大学 & 0.410 & 中央大学 & 0.246 & キリスト & 0.808\\
東京農業大学 & 0.410 & 明治大学 & 0.232 & 教会 & 0.774\\
オックスフォード大学 & 0.410 & 日本女子大学 & 0.231 & 山手 & 0.688\\ \hline
\end{tabular}
\label{table:wvg}
\end{table}


\subsection{WV\_ H}
単語ベクトルの次元数が100次元,反復回数が100,windowサイズを1000で生成したモデルの結果を表 \ref{table:wvh}に示す.

青山学院大学に近い大学として,北里大学や東京外国語大学など以外は総合大学が得られた.
また,その中でも同じミッション系の大学として,明治学院大学や立教大学,同志社大学,上智大学も得られた.

一方で,$ 青山学院大学 - キリスト教 $ でミッション系の大学の同志社大学や,オックスフォード大学などの大学が得られた.

青山学院大学と明治学院大学で共通の近い単語では,ミッション系の大学に関する単語や神学という単語とともに,新しく統合という単語が得られた.
この単語は神学部や専門部の統合という歴史的な背景から得られたと考えられる.

\begin{table}[H]
\caption{WV\_ Hの検証結果}
\centering
\footnotesize
\begin{tabular}{ll|ll|ll}
\hline
\multicolumn{2}{c}{青山学院大学に近い大学} & \multicolumn{2}{c}{青山学院大学 - キリスト教} & \multicolumn{2}{c}{青山学院大学と明治学院大学で共通の近い単語}
% \multicolumn{2}{c}{} & \\ \hline
\\ \hline
大学名 & 類似度 & 大学名 & 類似度 & 単語 & 類似度の合計
\\ \hline \hline
明治大学 & 0.494 & 明治大学 & 0.263 & キリスト教 & 0.840\\
明治学院大学 & 0.457 & 北里大学 & 0.255 & イギリス & 0.764\\
立教大学 & 0.435 & 東京外国語大学 & 0.230 & キリスト & 0.703\\
同志社大学 & 0.429 & 同志社大学 & 0.225 & 教会 & 0.671\\
北里大学 & 0.428 & 学習院大学 & 0.219 & 前期 & 0.661\\
東京外国語大学 & 0.424 & オックスフォード大学 & 0.197 & 神学 & 0.655\\
関西学院大学 & 0.377 & 東北大学 & 0.189 & 統合 & 0.613\\
上智大学 & 0.344 & 名古屋大学 & 0.185 & 併設 & 0.608\\
中央大学 & 0.337 & 九州大学 & 0.175 & 山手 & 0.606\\
法政大学 & 0.334 & 関西学院大学 & 0.156 & チャペル & 0.605\\ \hline
\end{tabular}
\label{table:wvh}
\end{table}