\section{GloVeのモデル検証結果}
本節ではGloVeのモデル検証結果をまとめた.
GloVeで生成したモデル名と,対応するパラメータの一覧を表 \ref{table:gvResultAll}に示す.

\begin{table}[htbp]
\caption{GloVeパラメータの詳細}
\centering
\begin{tabular}{llll}
\hline
モデル名 & 単語ベクトルの次元数 & 反復回数 & windowサイズ
\\ \hline \hline
GV\_ A & 50 & 10 & 10\\ \hline
GV\_ B & 100 & 10 & 10\\ \hline
GV\_ C & 50 & 100 & 10\\ \hline
GV\_ D & 100 & 100 & 10\\ \hline
GV\_ E & 50 & 10 & 1000\\ \hline
GV\_ F & 100 & 10 & 1000\\ \hline
GV\_ G & 50 & 100 & 1000\\ \hline
GV\_ H & 100 & 100 & 1000\\ \hline
\end{tabular}
\label{table:gvResultAll}
\end{table}

\subsection{GV\_ A}
単語ベクトルの次元数が50次元,反復回数が10,windowサイズを10で生成したモデルの結果を\ref{table:gva}に示す.

このモデルからは,青山学院大学に近い大学として中央大学や立教大学,上智大学,明治大学,法政大学などの大学が得られた.
これらの大学は総合大学である点や,ミッション系大学である点などの類似点がある.
さらに,学習したデータでは考慮されなかった偏差値の近さも表現されている.

また,$ 青山学院大学 - キリスト教 $ では,上位10校からはミッション系の大学が出現しなかった.
しかし東京工業大学や大阪体育大学などは関連性が低いと考えられる.
国際連合大学は大学ではないが,大学院は存在し青山学院大学と立地は非常に近いため出現したと考えられる.

青山学院大学と明治学院大学で共通の近い単語に関しては,学習不足であった.

\begin{table}[H]
\caption{GV\_ Aの検証結果}
\centering
\footnotesize
\begin{tabular}{ll|ll|ll}
\hline
\multicolumn{2}{c}{青山学院大学に近い大学} & \multicolumn{2}{c}{青山学院大学 - キリスト教} & \multicolumn{2}{c}{青山学院大学と明治学院大学で共通の近い単語}
% \multicolumn{2}{c}{} & \\ \hline
\\ \hline
大学名 & 類似度 & 大学名 & 類似度 & 単語 & 類似度の合計
\\ \hline \hline
中央大学 & 0.764 & 中央大学 & 0.559 & 学位 & 0.922\\
立教大学 & 0.747 & 東京工芸大学 & 0.557 & & \\
上智大学 & 0.71 & 立教大学 & 0.537 & & \\
駒澤大学 & 0.71 & 法政大学 & 0.506 & & \\
明治大学 & 0.683 & 国際連合大学 & 0.499 & & \\
学習院大学 & 0.672 & 関西大学 & 0.482 & & \\
法政大学 & 0.671 & 明治大学 & 0.482 & & \\
実践女子大学 & 0.623 & 大阪芸術大学 & 0.472 & & \\
東京理科大学 & 0.616 & 大阪体育大学 & 0.471 & & \\
成蹊大学 & 0.609 & 金沢医科大学 & 0.461 & & \\ \hline
\end{tabular}
\label{table:gva}
\end{table}


\subsection{GV\_ B}
単語ベクトルの次元数が100次元,反復回数が10,windowサイズを10で生成したモデルの結果を\ref{table:gvb}に示す.

青山学院大学に近い大学として,Word2Vecの方では見られなかった愛知淑徳大学や大阪体育大学,芝浦工業大学などが得られた.
これらの大学が学部構成が大きく異なる上,ミッション系の大学ではないためモデルの精度としては妥当性が低いと言える.

$ 青山学院大学 - キリスト教 $ の結果から,中央大学を除いて関連性の低い大学が出現した.

このモデルでは学習不足により青山学院大学と明治学院大学で共通の近い単語が得られなかった.


\begin{table}[H]
\caption{GV\_ Bの検証結果}
\centering
\footnotesize
\begin{tabular}{ll|ll|ll}
\hline
\multicolumn{2}{c}{青山学院大学に近い大学} & \multicolumn{2}{c}{青山学院大学 - キリスト教} & \multicolumn{2}{c}{青山学院大学と明治学院大学で共通の近い単語}
% \multicolumn{2}{c}{} & \\ \hline
\\ \hline
大学名 & 類似度 & 大学名 & 類似度 & 単語 & 類似度の合計
\\ \hline \hline
中央大学 & 0.639 & 東京工芸大学 & 0.52 & & \\
立教大学 & 0.588 & 中央大学 & 0.471 & & \\
学習院大学 & 0.562 & 大阪体育大学 & 0.46 & & \\
大東文化大学 & 0.551 & 大東文化大学 & 0.449 & & \\
明治大学 & 0.545 & 大阪芸術大学 & 0.435 & & \\
上智大学 & 0.537 & 神戸市外国語大学 & 0.415 & & \\
愛知淑徳大学 & 0.532 & 金沢医科大学 & 0.412 & & \\
大阪体育大学 & 0.53 & 京都女子大学 & 0.408 & & \\
芝浦工業大学 & 0.513 & 名城大学 & 0.407 & & \\
駒澤大学 & 0.496 & 会津大学 & 0.396 & & \\ \hline
\end{tabular}
\label{table:gvb}
\end{table}

\subsection{GV\_ C}
単語ベクトルの次元数が50次元,反復回数が100,windowサイズを10で生成したモデルの結果を\ref{table:gvc}に示す.

% 青山学院大学に近い大学として得られた大学は偏差値,立地の観点から妥当なものと考えられる.
青山学院大学に近い大学として得られた大学は,学部構成などの観点から妥当性の高い結果となっていると考えられる.
しかし $ 青山学院大学 - キリスト教 $ から,上智大学と立教大学が得られた.
これらの大学はミッション系であるため,モデルの妥当性は低いと考えられる.
また,このモデルでは学習不足により青山学院大学と明治学院大学で共通の近い単語が得られなかった.

\begin{table}[H]
\caption{GV\_ Cの検証結果}
\centering
\footnotesize
\begin{tabular}{ll|ll|ll}
\hline
\multicolumn{2}{c}{青山学院大学に近い大学} & \multicolumn{2}{c}{青山学院大学 - キリスト教} & \multicolumn{2}{c}{青山学院大学と明治学院大学で共通の近い単語}
% \multicolumn{2}{c}{} & \\ \hline
\\ \hline
大学名 & 類似度 & 大学名 & 類似度 & 単語 & 類似度の合計
\\ \hline \hline
立教大学 & 0.691 & 中央大学 & 0.476 & & \\
中央大学 & 0.667 & 明治大学 & 0.466 & & \\
上智大学 & 0.632 & 成蹊大学 & 0.437 & & \\
明治大学 & 0.616 & 法政大学 & 0.426 & & \\
駒澤大学 & 0.612 & 武蔵大学 & 0.407 & & \\
法政大学 & 0.601 & 上智大学 & 0.404 & & \\
学習院大学 & 0.563 & 駒澤大学 & 0.391 & & \\
早稲田大学 & 0.497 & 立教大学 & 0.366 & & \\
成蹊大学 & 0.479 & 福岡大学 & 0.354 & & \\
東京理科大学 & 0.476 & 成城大学 & 0.347 & & \\ \hline
\end{tabular}
\label{table:gvc}
\end{table}

\subsection{GV\_ D}
単語ベクトルの次元数が100次元,反復回数が100,windowサイズを10で生成したモデルの結果を\ref{table:gvd}に示す.

% 青山学院大学に近い大学として新しく大東文化大学が得られた.
% しかし評価項目を考慮すると,法政大学や上智大学よりも類似度が高い結果は妥当性が低いと考えられる.
青山学院大学に近い大学として得られた大学は学部構成の観点から近いものであると考えられる.
しかし芝浦工業大学は単科大学であるため,学部構成の観点から遠いと言える.

$ 青山学院大学 - キリスト教 $ の結果から,淑徳大学が新しく得られた.
キリスト教の要素を除いた結果,仏教系の大学が出現した様に見えるが,学部構成的に類似しているとは言い難い.

また,このモデルでは学習不足により青山学院大学と明治学院大学で共通の近い単語が得られなかった.

\begin{table}[H]
\caption{GV\_ Dの検証結果}
\centering
\footnotesize
\begin{tabular}{ll|ll|ll}
\hline
\multicolumn{2}{c}{青山学院大学に近い大学} & \multicolumn{2}{c}{青山学院大学 - キリスト教} & \multicolumn{2}{c}{青山学院大学と明治学院大学で共通の近い単語}
% \multicolumn{2}{c}{} & \\ \hline
\\ \hline
大学名 & 類似度 & 大学名 & 類似度 & 単語 & 類似度の合計
\\ \hline \hline
中央大学 & 0.642 & 中央大学 & 0.55 & & \\
明治大学 & 0.549 & 法政大学 & 0.412 & & \\
立教大学 & 0.542 & 明治大学 & 0.377 & & \\
学習院大学 & 0.539 & 関西大学 & 0.375 & & \\
大東文化大学 & 0.52 & 淑徳大学 & 0.37 & & \\
上智大学 & 0.52 & 武蔵大学 & 0.359 & & \\
成城大学 & 0.509 & 東洋大学 & 0.355 & & \\
駒澤大学 & 0.486 & 東京理科大学 & 0.349 & & \\
法政大学 & 0.474 & 駒澤大学 & 0.348 & & \\
芝浦工業大学 & 0.455 & 東京工科大学 & 0.348 & & \\ \hline
\end{tabular}
\label{table:gvd}
\end{table}

\subsection{GV\_ E}
単語ベクトルの次元数が50次元,反復回数が10,windowサイズを1000で生成したモデルの結果を\ref{table:gve}に示す.

青山学院大学に近い大学からは,女子美術大学がもっとも類似度が高い大学として得られた.
学習したデータにキャンパスの所在地は考慮されていないが,相模原という単語に影響を受けた可能性がある.

$ 青山学院大学 - キリスト教 $ の結果からは相模女子大学が得られた.
こちらも左の結果と同様に相模原という単語に影響を受けた可能性がある.
また,このモデルの結果からは単科大学が出力される傾向が強かった.
そのため,モデルの精度としては低いと考えられる.

青山学院大学と明治学院大学で共通の近い単語は,ミッション系の大学を連想する単語が出現したものの,類似度の合計値が高い単語は関連性が不明確なものであった.

\begin{table}[H]
\caption{GV\_ Eの検証結果}
\centering
\footnotesize
\begin{tabular}{ll|ll|ll}
\hline
\multicolumn{2}{c}{青山学院大学に近い大学} & \multicolumn{2}{c}{青山学院大学 - キリスト教} & \multicolumn{2}{c}{青山学院大学と明治学院大学で共通の近い単語}
% \multicolumn{2}{c}{} & \\ \hline
\\ \hline
大学名 & 類似度 & 大学名 & 類似度 & 単語 & 類似度の合計
\\ \hline \hline
女子美術大学 & 0.647 & 相模女子大学 & 0.442 & 芸術 & 1.024\\
立教大学 & 0.631 & 神奈川工科大学 & 0.396 & コミュニティ & 0.93\\
聖心女子大学 & 0.622 & 芝浦工業大学 & 0.391 & 音楽 & 0.925\\
東京工業大学 & 0.615 & 北海道教育大学 & 0.389 & 学位 & 0.874\\
津田塾大学 & 0.613 & 女子美術大学 & 0.373 & 礼拝 & 0.854\\
上智大学 & 0.611 & 東京農業大学 & 0.372 & 心理 & 0.854\\
関西学院大学 & 0.61 & 国際連合大学 & 0.362 & 合同 & 0.825\\
明治学院大学 & 0.605 & 目白大学 & 0.362 & & \\
帝京大学 & 0.602 & 創価大学 & 0.359 & & \\
東京農業大学 & 0.599 & 帝京大学 & 0.351 & & \\ \hline
\end{tabular}
\label{table:gve}
\end{table}

\subsection{GV\_ F}
単語ベクトルの次元数が100次元,反復回数が10,windowサイズを1000で生成したモデルの結果を\ref{table:gvf}に示す.

青山学院大学に近い大学からは,関西学院大学や聖心女子大学,明治学院大学などのミッション系の大学が得られた.
聖心女子大学や東京工業大学,女子美術大学等は単科大学なので,学部構成の観点から類似度は高くないはずである.

$ 青山学院大学 - キリスト教 $ の結果からは,このモデルでも単科大学の大学が上位に表示された.

また,青山学院大学と明治学院大学で共通の近い単語からは,関係性のある単語が取得できなかった.

\begin{table}[H]
\caption{GV\_ Fの検証結果}
\centering
\footnotesize
\begin{tabular}{ll|ll|ll}
\hline
\multicolumn{2}{c}{青山学院大学に近い大学} & \multicolumn{2}{c}{青山学院大学 - キリスト教} & \multicolumn{2}{c}{青山学院大学と明治学院大学で共通の近い単語}
% \multicolumn{2}{c}{} & \\ \hline
\\ \hline
大学名 & 類似度 & 大学名 & 類似度 & 単語 & 類似度の合計
\\ \hline \hline
関西学院大学 & 0.541 & 神奈川工科大学 & 0.344 & 女学院 & 0.753\\
聖心女子大学 & 0.525 & 相模女子大学 & 0.334 & 英和 & 0.742\\
早稲田大学 & 0.514 & 北海道教育大学 & 0.333 & 芸術 & 0.731\\
明治学院大学 & 0.513 & 目白大学 & 0.315 & 前期 & 0.705\\
実践女子大学 & 0.503 & 国際連合大学 & 0.315 & 音楽 & 0.676\\
東京工業大学 & 0.495 & 関西学院大学 & 0.299 & コミュニティ & 0.674\\
女子美術大学 & 0.492 & 関東学院大学 & 0.293 & 貢献 & 0.631\\
津田塾大学 & 0.489 & 会津大学 & 0.285 & & \\
帝京大学 & 0.485 & 大妻女子大学 & 0.283 & & \\
東京大学 & 0.474 & 聖心女子大学 & 0.279 & & \\ \hline
\end{tabular}
\label{table:gvf}
\end{table}


\subsection{GV\_ G}
単語ベクトルの次元数が50次元,反復回数が100,windowサイズを1000で生成したモデルの結果を\ref{table:gvg}に示す.

このモデルでも青山学院大学に近い大学として関西学院大学が得られた.
しかし女子美術大学や国際大学,愛知淑徳大学などは関連性が見出せなかった.

$ 青山学院大学 - キリスト教 $ の結果からは,単科大学の出現が多かった.

また,青山学院大学と明治学院大学で共通の近い単語からは,関係性のある単語が取得できなかった.

\begin{table}[H]
\caption{GV\_ Gの検証結果}
\centering
\footnotesize
\begin{tabular}{ll|ll|ll}
\hline
\multicolumn{2}{c}{青山学院大学に近い大学} & \multicolumn{2}{c}{青山学院大学 - キリスト教} & \multicolumn{2}{c}{青山学院大学と明治学院大学で共通の近い単語}
% \multicolumn{2}{c}{} & \\ \hline
\\ \hline
大学名 & 類似度 & 大学名 & 類似度 & 単語 & 類似度の合計
\\ \hline \hline
関西学院大学 & 0.505 & 神奈川工科大学 & 0.41 & 女学院 & 0.63\\
女子美術大学 & 0.497 & 東京農業大学 & 0.38 & 芸術 & 0.612\\
聖心女子大学 & 0.47 & 明治大学 & 0.377 & 英和 & 0.572\\
立教大学 & 0.459 & 相模女子大学 & 0.372 & 学位 & 0.485\\
国際大学 & 0.456 & 東京情報大学 & 0.369 & & \\
相模女子大学 & 0.445 & 帝京大学 & 0.355 & & \\
同志社大学 & 0.445 & 神奈川大学 & 0.35 & & \\
津田塾大学 & 0.425 & 帝京平成大学 & 0.347 & & \\
愛知淑徳大学 & 0.421 & 東京農工大学 & 0.333 & & \\
東京情報大学 & 0.411 & 福島県立医科大学 & 0.328 & & \\ \hline
\end{tabular}
\label{table:gvg}
\end{table}

\subsection{GV\_ H}
単語ベクトルの次元数が100次元,反復回数が100,windowサイズを1000で生成したモデルの結果を\ref{table:gvh}に示す.

このモデルでは,青山学院大学にもっとも近い大学として,関西学院大学が得られた.
また青山学院女子短期大学やオックスフォード大学など,他のモデルではあまり見られなかった大学が得られた.

$ 青山学院大学 - キリスト教 $ の結果からは,もっとも近い大学として関西学院大学が得られたが,ミッション系の大学であるため,モデルの精度は妥当ではないと考えられる.

青山学院大学と明治学院大学で共通の近い単語からは,ミッション系の大学を連想する単語と,統合という歴史的な背景を連想する単語が得られた.

\begin{table}[H]
\caption{GV\_ Hの検証結果}
\centering
\footnotesize
\begin{tabular}{ll|ll|ll}
\hline
\multicolumn{2}{c}{青山学院大学に近い大学} & \multicolumn{2}{c}{青山学院大学 - キリスト教} & \multicolumn{2}{c}{青山学院大学と明治学院大学で共通の近い単語}
% \multicolumn{2}{c}{} & \\ \hline
\\ \hline
大学名 & 類似度 & 大学名 & 類似度 & 単語 & 類似度の合計
\\ \hline \hline
関西学院大学 & 0.353 & 関西学院大学 & 0.315 & 聖書 & 0.471\\
東京外国語大学 & 0.322 & 東京農業大学 & 0.272 & 定期 & 0.458\\
東京農業大学 & 0.271 & 神戸大学 & 0.214 & 統合 & 0.455\\
静岡大学 & 0.27 & 東京理科大学 & 0.213 & 女学院 & 0.451\\
青山学院女子短期大学 & 0.257 & 横浜市立大学 & 0.213 & 英和 & 0.426\\
オックスフォード大学 & 0.242 & 静岡大学 & 0.18 & チャペル & 0.369\\
神戸大学 & 0.235 & 昭和女子大学 & 0.179 & 前期 & 0.365\\
東京工業大学 & 0.229 & 北海道大学 & 0.179 & キリスト & 0.359\\
九州大学 & 0.229 & 明治大学 & 0.175 & キリスト教 & 0.357\\
筑波大学 & 0.221 & 九州大学 & 0.175 & 申し出 & 0.351\\ \hline
\end{tabular}
\label{table:gvh}
\end{table}

