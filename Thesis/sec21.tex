\section{歩行者検出について}
現在一般的に用いられるオブジェクト検出技術の多くは,機械学習を元に対象オブジェクトを学習して検出する手法が一般的である.
ここでは先行研究で用いられたフレーム間差分法と本研究で用いた機械学習について述べる.


\subsection{フレーム間差分}

\LaTeX を利用する際には,HTMLの様なマークアップを文章中に記述する.

適切なマークアップさえすれば,その構造に応じて書式を整形して出力することができる.
また,論文などの文章を書く際の煩雑な手間を,大幅に削減することができる.
その例を一部列挙する.
\begin{itemize}
\item 章や節などの見出しの書式設定は,自動
\item 目次ページ番号,参考文献番号の付加や引用表示,図表や数式の番号割り振りや引用表示が自動
\item 数式が非常にきれいに表現できる
\end{itemize}

その一方で以下の欠点も存在している.
\begin{itemize}
\item \LaTeX が使えるようにソフトウェアを導入しなければならない
\item 最低限のマークアップを憶えなければならない
\item マークアップ以外の命令も憶えなければならない
\end{itemize}


\subsection{デメリットを解決する本文書と克服するために}
デメリットを解決するために,宮治研用のスタイルパッケージを整え,本文書を作成した.

\begin{itemize}
\item 「最低限のマークアップ」本文書のサンプルを参考にマネをすれば,完璧に憶える必要は無い
\item 「マークアップ以外の命令」自動実行するバッチファイルを準備したため,これを実行するだけで良い
\end{itemize}

よって,\LaTeX の環境を自分のパソコンに整えさえすれば,比較的容易に論文作成ができるであろう.

Macintosh への \LaTeX のインストールは,奥村他有志による TeX Wiki の 「MacTeX のインストール」\cite{mactex} を参考にすると良いだろう.
また,Windows へのインストールは,奥村他有志による TeX Wiki の 「W32Tex」\cite{w32tex} を参考にすると良いだろう.
