
近年,大学進学という選択肢は高校生にとって一般的な選択肢として受け入れられるようになった.
また2020年4月から高等教育の修学支援制度が施行される予定であり,この制度の施行により大学進学率は増加すると考えられる.
さらに新設大学,新設学部も年々増加しているなかで,特定の大学に志願者が集中するという問題が提起されている.

本研究では,高校生が自分にあった適切な大学を選択できるように支援するシステムを考案した.
ネットの大学に関する記事から大学ごとの単語ベクトルを生成し,大学間の類似度を測る.
また,キャンパスの所在地や学部ごとの偏差値といったデータを利用して,大学間の潜在的な関係性の可視化を支援するシステムを構築した.

検証の結果,特定の大学に絞ったシステムを構築し,潜在的な関係性の可視化を実現した.
また,より横断的なデータを利用したシステムの精度向上の可能性に関して示した.
