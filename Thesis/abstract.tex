高校生にとって大学進学という選択肢は,一般的なものとして受け入れられるようになった.
その一方で,特定の大学に志願者が集中するという問題が生じている.
この問題に対し,政府は大学への入学者数を厳格に制限する対策を施行し,大学の定員割れを減少させるなどの一定の効果をあげることができた.
しかし,これは単に大学の志望先を,より下位の偏差値グループに移動させだだけと考えることができ,大学選びの根源的な部分に影響するものではない。
受験生が単に「自分が合格できる偏差値の大学学部である」ことや「自宅から近い」といった画一的な判断基準で大学を選択していることこそが問題といえる.
それを解決する方法のひとつとして、受験生が大学選択の参考となる情報を提示したり、似た大学を比較・認知できるようなデータを示すことが重要であると考えた。

以上の背景から本研究では,受験生の大学選択の情報支援を目的とし,大学の多面的・特徴的な情報を示し,大学同士の比較ができる支援するシステムを考案し開発した.
システムの根幹となる大学の特徴や比較のための仕組みは,大学に関わる単語ベクトルモデルを機械学習によってつくり,それをAPIによるサービスとして情報が引き出せるようにした.
単語ベクトルを扱うことによって,大学名を元に類似度の比較したり,演算(単語の足し引き)による情報提示が可能になる.
この単語ベクトルは,Wikipedia の各大学毎の情報,大学プレスセンターに掲載されている各大学に関する記事,パスナビに記載されている情報をクローリングとスクレイピングによって収集し,学習して作成した.
学習は,Word2VecとGloVeの手法にて作成・比較した.また,それぞれの手法での学習パラメータを変更して学習・比較した.
さらに,でき上がった単語ベクトルモデルからの出力に対し,大学学部ごとの偏差値やキャンパス所在地の緯度経度といった情報を補助的に用いて,大学間の関係性を示したり絞り込みができるインターフェースを作成した.

また,システムの根幹である単語ベクトルのモデルについて,生成したモデルの精度を比較する検証実験をおこなった,
検証実験では,学習手法とパラメータを変化させながら,「青山学院大学と類似度が近い大学」「青山学院大学 $--$ キリスト教 演算の結果(大学名)」「青山学院大学と明治学院大学で共通の近い意味を持つ単語」の3つのタスクの出力結果を目視で評価した.
共通の評価基準で比較した結果,各々のタスクに対して適切に出力する学習手法とパラメータの組み合わせを得ることができた.