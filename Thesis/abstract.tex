
% 近年,大学進学という選択肢は高校生にとって一般的な選択肢として受け入れられるようになった.
% また2020年4月から高等教育の修学支援制度が施行される予定であり,この制度の施行により大学進学率は増加すると考えられる.
% しかし進学率増加に伴い,特定の大学に志願者が集中するという問題が提起されている.
% この問題に対し,政府は入学者数を厳格に制限する対策を施行して一定の効果が見られたが,これは一時的なものであり,根本的な解決策としては限界がある.
% 志願者が特定の大学に集中する問題の本質的な問題点は,受験生が「家から近い」や「自分の力にあってたから」と言った画一的な判断基準で大学を決めていることに起因する.

% そのため本研究では,高校生が自分にあった適切な大学を選択できるように支援するシステムを考案した.
% まず大学プレスセンターに掲載されている各大学に関する記事とWikipedia,パスナビの情報から単語ベクトルを生成することで,各大学名を単語ベクトルで表現する.
% 大学名をベクトルで表現することで,大学名を元に類似度の比較や単語の足し引きが可能になる.
% 生成する単語ベクトルのモデルはWord2VecとGloVeで学習し,それぞれの手法で学習する際のパラメータを調整してモデルを生成した.
% また,キャンパスの所在地や学部ごとの偏差値といったデータを利用して,大学間の潜在的な関係性の可視化を支援するシステムを構築した.

% 構築したシステムで利用する単語ベクトルのモデルについて,生成したモデルの精度を比較する検証実験をおこない,タスクごとに.
% 検証実験では,青山学院大学と類似度が近い大学,$ 青山学院大学 - キリスト教 $ で得られる大学,青山学院大学と明治学院大学で共通の近い意味を持つ単語を取得する,の3つのタスクの出力を目視で評価した.
% それぞれのタスクで評価基準を設定して検証した結果,全てのタスクに対してロバストなモデルと,各々のタスクに対して最も適切なモデルを得た.

% 検証実験の結果から,タスクに応じたモデルをシステムに利用することで出力結果の妥当性を向上させた.
% また,データセットの都合上,対象の大学を関東近郊の21校に限定し,Web APIで実装することで,様々な環境下で単語ベクトルのモデルを利用できるシステムを構築した.

近年,大学進学という選択肢は高校生にとって一般的な選択肢として受け入れられるようになった.
また2020年4月から高等教育の修学支援制度が施行される予定であり,この制度の施行により大学進学率は増加すると考えられる.
しかし進学率増加に伴い,特定の大学に志願者が集中するという問題が提起されている.
この問題に対し,政府は入学者数を厳格に制限する対策を施行して一定の効果が見られたが,これは一時的なものであり,根本的な解決策としては限界がある.
志願者が特定の大学に集中する問題の本質的な問題点は,受験生が「家から近い」や「自分の力にあってたから」と言った画一的な判断基準で大学を決めていることに起因する.

そのため本研究では,大学に関するWikipediaの記事と大学プレスセンターの記事をデータセットとして,単語ベクトルを学習しモデルを生成した.
また,パスナビの大学のページから取得した学部ごとのキャンパスや偏差値の情報と,キャンパス所在地の緯度経度といった情報を補助的に用いて,大学間の関係性を可視化するための支援をするシステムを考案した.

検証実験から,特定のタスクに対して精度の高いモデルを評価して,タスクに応じたモデルを利用することで出力結果の妥当性を向上させた.

データセットの都合上,対象の大学は関東近郊の21校に絞ったが,Web API形式で実装することで様々な環境で単語ベクトルのモデルを利用できるシステムを構築できた.