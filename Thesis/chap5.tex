\chapter{おわりに}
本研究では,大学に関するWikipediaの記事と大学プレスセンターの記事をデータセットとして,単語ベクトルを学習しモデルを生成した.
また,パスナビの大学のページから取得した学部ごとのキャンパスや偏差値の情報と,キャンパス所在地の緯度経度といった情報を補助的に用いて,大学間の関係性を可視化するための支援をするシステムを考案した.

検証実験から,特定のタスクに対して精度の高いモデルを評価して,タスクに応じたモデルを利用することで出力結果の妥当性を向上させた.
データセットの都合上,対象の大学は関東近郊の21校に絞ったが,Web API形式で実装することで様々な環境で単語ベクトルのモデルを利用できるシステムを構築した.

\section{改善点}
まずデータセットに偏りがあることが改善点として挙げられる.
今回使用したデータは,Wikipedia,パスナビ,大学プレスセンターの記事をマージしたものであるが,大学プレスセンターの記事がデータセットにおけるほとんどの割合を占めている.
そこで大学プレスセンターの記事に偏りがあると,モデルの精度に影響が出る.
さらに記事の数も大学によって差があるため,安定したデータセットを収集する必要がある.
しかし日本全国の全ての大学を考慮に入れる場合,有名大学と新設大学では利用できるデータの量に大きな差がある.

また暗黙知のように我々が知っている情報がインターネットから取得できない点も挙げられる.
例えば大学によっておしゃれな印象であったり,お金持ちが多い大学などといった情報は収集するのが難しい.
このような情報はTwitterや5ちゃんねるなどから取得できる可能性があるが,ノイズが非常に多いことが予想される.
それゆえ,様々な媒体から横断的に大量のデータを収集する必要がある.