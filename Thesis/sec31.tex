\section{ユースケース}
本節では,本研究で構築したシステムのユースケースについて説明する.

まず本システムは,ある大学に関連する大学をユーザに推薦する機能を提供する.
具体的には,システム開発者に向けて,Web API形式で関連大学の推薦機能を提供するものである.

想定されるエンドユーザは大学受験を控えた高校生とする.
本研究で構築したシステムが提供する機能を組み込んだWebアプリケーションを利用することで,漠然と気になっている大学から関連する大学を検索することができる.
% エンドユーザは検索ボックスに大学名を入力する.
類似度の高い大学を検索すると,あらかじめ学習した単語ベクトルから,コサイン類似度が最も近い大学が順番に表示される.
表示された大学との共通の近い単語を検索すると,どのような単語を通して検索元の大学と近いのかを確認することができる.

また検索結果の大学から気になる学部を選択すると,偏差値の近い大学の候補が表示される.
この時,キャンパスの立地も考慮して検索結果をフィルタリングすることもできる.

さらに,大学名をベクトルで表現しているため,気になっている大学に対して何かしらの単語を足し算したり,引き算したりすることが可能である.
