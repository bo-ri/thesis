\section{本研究で構築したシステム}
本節では,本研究で構築したシステムの概要とについて説明する.
\subsection{概要}
本研究では,単語ベクトルと大学に関する情報を用いた大学名の検索機能をWeb API形式で利用できる形で実装した.
大学に関する情報は,学部ごとの偏差値とキャンパス名,大学ごとのキャンパスの緯度経度を用いる.
% キャンパスの所在地を緯度と経度で利用したのは,キャンパス間の距離を2次元で表現できるためである.

各部は単語ベクトルのモデルを提供するモデル部,学部・偏差値データとキャンパス所在地を提供する大学情報データベース部から構築されている.


\subsection{モデル部}
本システムにおけるモデル部について説明する.
モデル部はWord2VecとGloVeを用いて学習した単語ベクトルの機能を提供するものである.
大学名と,それに関連する単語をベクトルで表現することで,大学から特定の特徴を加算・減算することができる.
また,ベクトル間のコサイン類似度を計算することで,ある大学の単語ベクトルと近いベクトルで表現された大学を取得できる.

本システムで実装した主な機能は,大学名から近い大学を取得する機能,大学に要素を加算する機能,大学から要素を減算する機能,ある大学と他の大学間で共通の近い意味を持った単語の検索機能である.
各機能の詳細に関しては次節で説明する.

また本研究では,検索対象の大学を関東近郊の特定の大学に制限した.
制限した理由は,単語ベクトルを学習するのに十分なデータを用意することが困難であったためである.
そのため,本研究では比較的データが入手できる21校に絞ってシステムを構築した.
対象の大学は表 \ref{table:univs}に示す.
本研究で生成した単語ベクトルの学習に利用したデータは主に3種類挙げられる.
1つは対象の大学それぞれのWikipediaの記事,2つ目はパスナビの各大学のページから沿革,LIFE\&STUDY,大学院・研究室,3つ目は大学プレスセンターから各大学名で検索した結果の記事である.

\begin{table}[htbp]
\caption{対象の大学}
\centering
\begin{tabular}{|lllll|}
\hline
% モデル名 & 単語ベクトルの次元数 & 反復回数 & windowサイズ
% \\ \hline \hline
青山学院大学 & 中央大学 & 立教大学 & 法政大学 & 明治大学\\
早稲田大学 & 慶應義塾大学 & 上智大学 & 国際基督教大学 & \\
日本大学 & 東洋大学 & 駒澤大学 & 専修大学 & \\
成蹊大学 & 成城大学 & 明治学院大学 & 学習院大学 & \\
獨協大学 & 國學院大学 & 武蔵大学 & 東京理科大学 & \\  \hline
\end{tabular}
\label{table:univs}
\end{table}

% \subsection{ストップワード}
\subsection{大学情報データベース部}
大学情報データベース部は本システムで対象を絞った21校の大学に関するデータを格納している.
データのフォーマットはJSON形式で提供される.
データの内容は,大学毎にキャンパスの情報があり,キャンパス情報の中に学部と対応した偏差値とキャンパスの所在地の情報が存在する.

学部と対応した偏差値のデータは,パスナビから取得した偏差値を利用した.
偏差値が区間で提供されていた場合は,下限値と上限値の平均を採用した.

所在地はパスナビから取得したキャンパスの住所を元に,キャンパス所在地の緯度と経度を利用した.
緯度と経度を採用した理由として,キャンパス間の距離を2次元で比較するためである.

これらのデータを元に,偏差値の近い大学や,立地の近い大学を学部毎に比較することができる.
